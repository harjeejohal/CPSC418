\documentclass[11pt]{article}

\usepackage{amsthm,amsmath,amssymb,amsfonts}
\usepackage[margin=1in]{geometry}

\parindent 0pt
\parskip 3mm

\theoremstyle{definition}
\newtheorem*{solution}{Solution}

\begin{document}

\begin{center}
{\bf \Large CPSC 418 / MATH 318 --- Introduction to Cryptography

ASSIGNMENT 1 }
\end{center}

\hrule 	

\textbf{Name:} Harjee Johal (replace by your name) \\
\textbf{Student ID:} 30000668 (replace by your ID number)

\medskip \hrule

\begin{enumerate} \itemsep 20pt

\item[] \textbf{Problem 1} --- Linear Feedback Shift Register Key Streams (10 marks)

\begin{enumerate}

\item The first 19 bits generated by this sequence are: $z_0 = 1, z_1 = 0, z_2 = 1, z_3 = 0, z_4 = 1, z_5 = 1, z_6 = 0, z_7 = 0, z_8 = 1, z_9 = 0, z_{10} = 0, z_{11} = 0, z_{12} = 1, z_{13} = 1, z_{14} = 1, z_{15} = 1, z_{16} = 0, z_{17} = 1, z_{18} = 0$\\

\item We can use linear recurrence (1) provided in the question, along with the known sequence: \\ \\
\centerline{$z_{i}, z_{i+1}, \ldots , z_{i+2m-1}$}
\\ \\
to generate the following system of \textit{m} equations:\\ \\
\begin{align*}
z_{i+m} &= c_{m-1} z_{i+m-1} + c_{m-2} z_{i+m-2} + \ldots + c_0 z_i\\
z_{i+m+1} &= c_{m-1} z_{i+m} + c_{m-2} z_{i+m-1} + \ldots + c_0 z_{i+1}\\
&\ldots \\
z_{i+2m-2} &= c_{m-1} z_{i+2m-3} + c_{m-2} z_{i+2m-4} + \ldots + c_0 z_{i+m}\\
z_{i+2m-1} &= c_{m-1} z_{i+2m-2} + c_{m-2} z_{i+2m-3} + \ldots + c_0 z_{i+m-1}\\
\end{align*}
\\ \\
Since there are \textit{m} unknown coefficients $c_0, c_1, \ldots , c_{m-1}$, and we have \textit{m} equations utilizating these coefficients, we can generate a \textit{m} x \textit{m} matrix to solve for these coefficients.\\

\item \emph {}
\end{enumerate}

\newpage

\item[] \textbf{Problem 2} --- Password Counts (20 marks plus 5 bonus marks)

\begin{enumerate}
\item The total number of passwords of length 8 that can be generated using these 94 printable characters is $94^8$

\item There are two scenarios in this question that need to be considered.\\
\\ \textbf{Case 1: The first four letters of the child's name are all lowercase:}\\
If the child's name is in all lowercase, then there are $26^3 * 366 = 6,432,816$ possible password candidates. We know that the first
four characters of the password are the first four letters of the child's name, and since the first one is known to be 'L', that leaves
the next three slots unknown. Each of these slots could be any of the lowercase letters of the alphabet, so there are $26^3$
possibilities.\\

We also know that the last four characters of the password are the date and month that the user was born in, in DDMM format. The number
of unique DDMM combinations is equal to the number of days in the year - there is one possible DDMM combination for each day. Since we know
that the user was born in 2008, that means that there are 366 unique DDMM possibilities, since 2008 was a leap year.\\

By multiplying the number of possibilities for the first four characters of the password by the number of possibilities for the second half of the password,
we get $26^3 * 366 = 6,432,816 $.\\

\textbf{Case 2: The first four letters of the child's name are all uppercase:}\\
The number of possibilities in this case is the same as case one: $26^3 * 366 = 6,432,816$.\\
This scenario is almost identical to case one, with the exception that we're working with uppercase letters instead of lowercase letters. However, since there
are an equal number of uppercase and lowercase letters in the alphabet, the calculations are the exact same.\\

Therefore, the total number of password candidates is simply the sum of the number of possibilities of each case: $6,432,816 + 6,432,816 = 12,865,632$ total candidates.\\

\item The total number of passwords of length 8 that have at least one numerical digit and at least one special character can be computed by finding the number of passwords that
violate this condition, and subtracting them from the total number of passwords of length 8.\\ 

First, we compute the total number of passwords of length 8: $94^8$.\\

Next, we find the number of passwords of length 8 without a numerical digit in them. Since there are 10 numerical digits, the total number of passwords of length 8 that use none of them is:
$(94-10)^8 = 84^8$.

Next, we find the number of passwords of length 8 without a special character in them. There are 32 special characters, so the number of passwords of length 8 
that don't use them is: $(94-32)^8 = 62^8$.\\

Within these two computed subsets of the total passwords of length 8, we've double-counted a specific category. 
Namely, we've double counted the total number of passwords of length 8 that have neither a numerical digit nor a special characer in them. 
Therefore, to avoid these subset being subtracted from the total number of passwords of length 8 twice, it must be added once. The total number of passwords of 
length 8 that don't use a numercal digit or a special character is $(94 - 32 - 10)^8 = 52^8$.\\ 

Combining all of this information together, we can compute the number of passwords of length 8 with at least one numerical digit and at least one special character as:\\ \\
\textit{Total passwords} - \textit{passwords with no digits} - \textit{passwords with no special characters} + \textit{passwords with neither digits nor special characters}.\\

Substituting in the values that have been computed, we get $94^8 - 84^8 - 62^8 - 52^8$ passwords of length 8 that have at least one numerical digit and at least one special character.\\

\item The percentage of 8-character passwords that satisfy the requirements of part (c) is simply a ratio of the value computed in the previous question, and the total number of passwords
of length 8:\\
\centerline{$(94^8 - 84^8 - 62^8 - 52^8)/94^8 = 0.5487698\ldots \approx 54.9\%.$}\\

Therefore, approximately 54.9\% of passwords satisfy the requirement in (c).\\

\item (\textbf{Bonus}) In order to find the number of passwords of length 8 with at least one numerical digit and at least one special character and at least one uppercase letter, it is easier to find
all of the passwords that violate this condition, and then subtract that from the total of passwords. We'll represent the number of passwords with at least one uppercase, at least one special character, and
at least one numerical digit as $n_{target}$.\\
The total number of passwords of length 8 using 94 printable characters is $94^8$. Let us represent this value as $n_{total}$.\\

Next, we must find the number of passwords of length 8 without any uppercase letters. There are 26 uppercase letters, so the total number of passwords of length 8 without an uppercase letter is $(94 - 26)^8 = 68^8$.
We'll represent this value as $n_{uppercase}$.\\

We must also find the number of passwords of length 8 without any special characters. This was computed in part (c) and found to be $62^8$. We'll represent this value as $n_{special}$.\\

The next value to calculate is the number of passwords of length 8 without any numerical digits. This was also computed in part (c) and found to be $84^8$. We'll represent this value as $n_{numerical}$.\\

Also similarly to part (c), if we subtract $n_{uppercase}, n_{special}$, and $n_{numerical}$ from $n_{total}$, we'll end up subtracting passwords that fall into two of these categories twice. Namely,
passwords that fall into both $n_{uppercase}$ and $n_{special}$, passwords that fall into both $n_{uppercase}$ and $n_{numerical}$, and passwords that fall into both $n_{special}$ and $n_{numerical}$ will end up
being subtracted twice. In order to counter this, each of these subsets must be added back. Thus, we must compute the total number of passwords that fall into each of these subsets.\\

The number of passwords of length 8 with neither an uppercase letter nor a special character is $(94 - 26 - 32)^8 = 34^8$. We'll represent this as $n_{uppercase, special}$.\\

The number of passwords of length 8 with neither an uppercase letter nor a numerical digit is $(94 - 26 - 10)^8 = 58^8$. We'll represent this as $n_{uppercase, numerical}$.\\

The number of passwords of length 8 with neither an special character nor a numerical digit is $(94 - 32 - 10)^8 = 52^8$. We'll represent this as $n_{special, numerical}$.\\

With this, we have now accounted for the passwords that fall into two categories. However, there is one more category of passwords that hasn't been considered yet - passwords that have neither an uppercase letter
nor a special character nor a numerical digit. Let us represent this value as $n_{uppercase,special,numerical}$. This subset is subtracted from the total count three times: once when $n_{uppercase}$ is subtracted from $n_{total}$,
once when $n_{special}$ is subtracted from $n_{total}$, and once when $n_{numerical}$ is subtracted from $n_{total}$.\\
$n_{uppercase,special,numerical}$ is also added back to $n_{total}$ three times: once when $n_{uppercase,special}$ is added back to $n_{total}$, once when $n_{uppercase,numerical}$ is added back to $n_{total}$, and
once when $n_{special,numerical}$ is added back to $n_{total}$. Since $n_{uppercase,special,numerical}$ been subtracted three times and added three times, this subset is still present in the count of passwords
with neither an uppercase letter nor a special character nor a numerical digit. Therefore, it must be subtracted one more time to be removed from this count.\\

The value of $n_{uppercase,special,numerical}$ is $(94 - 26 - 32 - 10)  = 26^8$.\\

Combining all of this information, we can compute $n_{target}$ as:
\begin{align*}
n_{target} &= n_{total} - n_{uppercase} - n_{special} - n_{numerical} + n_{uppercase,special} + n_{uppercase,numerical} + n_{special,numerical} - n_{uppercase,special,numerical}
&= 94^8 - 68^8 - 62^8 - 84^8 + 34^8 + 58^8 + 52^8 - 26^8
\end{align*}
\\Therefore, the value of $n_{target}$ is $94^8 - 68^8 - 62^8 - 84^8 + 34^8 + 58^8 + 52^8 - 26^8$.\\

\item If each permissable character in a password is chosen with equal likelihood, that means that each password in that password space is chosen with equal likelihood. Therefore,
the entropy of the password space is $log_2(n)$, where \textit{n} is the number of passwords in the password space.\\

\textbf{For part (a):} In part (a) we asserted that the number of passwords of length 8 was $94^8$. Therefore, the entropy of this password space is:

\begin{align*}
entropy &= log_2(94^8)\\
&= 52.43671\ldots \\
&\approx 52.4\\
\end{align*}

\textbf{For part (c):} In part (c) we asserted that the number of passwords of length 8 with at least one numerical digit and at least one special character was $(94^8 - 84^8 - 62^8 - 52^8)$.
Therefore, the entropy of this password space is:

\begin{align*}
entropy &= log_2(94^8 - 84^8 - 62^8 - 52^8)\\
&= 51.5709\ldots \\
&\approx 51.6\\
\end{align*}

\item Assuming that each password candidate is chosen with equal likelihood, we would need $2^{128}$ passwords in order to achieve an entropy of 128. Since we have 94 printable
characters to use in our passwords, that means we just have to determine \textit{x} such that $94^x = 2^{128}$. Solving this equation, we get:\\

\begin{align*}
94^x &= 2^{128}\\
log_2(94^x) &= log_2(2^{128})\\
log_2(94^x) &= 128\\
x log_2(94) &= 128\\
x &= 128 / log_2(94)\\
x &=  19.5283 \ldots \\
x&\approx 19.5.
\end{align*}

However, there's no such thing as a password that's 19.5 characters long. Therefore, we must take the ceiling of this value. We cannot take the floor, because a password length
of 19 would have a entropy below 128. Therefore the minimum password length that guarantees a password space with entropy 128 is 20.

\end{enumerate}

\newpage

\item[] \textbf{Problem 3} --- Probabilities of Non-Collisions (26 marks)

\begin{enumerate}
\item
\item
\item
\item
\item
\item
\item
\item
\end{enumerate}

\newpage

\item[] \textbf{Problem 4} --- Equiprobability maximizes entropy for two outcomes, 10 marks
\begin{enumerate}
\item

\item
\item
\end{enumerate}

\newpage

\item[] \textbf{Problem 5} --- Cryptanalysis of a class of linear ciphers, 34 marks)

\textbf{*** Remove the text for this problem if you don't attempt it. ***}

\begin{enumerate}
\item
\item
\item
\item
\item
\item
\item
\item
\end{enumerate}

\newpage

\item[] \textbf{Problem 7} --- Mixed Vigen\`ere cipher cryptanalysis, 10 marks

\textbf{*** Remove the text for this problem if you don't attempt it. ***}


\end{enumerate}

\end{document}
