\documentclass[11pt]{article}

\usepackage{amsthm,amsmath,amssymb,amsfonts}
\usepackage[margin=1in]{geometry}

\parindent 0pt
\parskip 3mm

\theoremstyle{definition}
\newtheorem*{solution}{Solution}
\newcounter{problem}

\begin{document}

\begin{center}
{\bf \Large CPSC 418 / MATH 318 --- Introduction to Cryptography

ASSIGNMENT 2 }
\end{center}

\hrule 	

\textbf{Name:} Harjee Johal \\
\textbf{Student ID:} 30000668 \\

\medskip \hrule

\begin{enumerate} \itemsep 20pt

\stepcounter{problem}
\item[] \textbf{Problem \theproblem} ---  Arithmetic in the AES {\sc MixColumns} operation (22
    marks)

\begin{enumerate}
\item \begin{enumerate}

\item (a) (i) For this question, we are asked to prove that in AES MIXCOLUMNS arithmetic multiplying any 4-byte vector by \textit{y} is a circular left shift by one byte. Suppose that \textit{a} is a 4-byte vector such that \textit{a = $(a_3, a_2, a_1, a_0)$}. Let $a(y) = a_3y^3 + a_2y^2 + a_1y + a_0$ be the polynomial representation of \textit{a}. Thus, $a(y) * y$ is:

\begin{align*}
    a(y)*y &= (a_3y^3 + a_2y^2 + a_1y + a_0)*y \\
    &= a_3y^4 + a_2y^3 + a_1y^2 + a_0y.
\end{align*}

Next, we must perform reduction of $a(y)*y$ modulo \textit{M(y)}. We are told that $M(y) = 0$ and that $M(y) = y^4 + 1$. From this, can determine that since $y^4+1 = 0$, then $y^4 = 1$. Applying this to $a(y)*y$, we get:

\begin{align*}
    a(y) * y &= a_3y^4 + a_2y^3 + a_1y^2 + a_0y \\
    &= a_3(1) + a_2y^3 + a_1y^2 + a_0y \\
    &= a_2y^3 + a_1y^2 + a_0y + a_3.
\end{align*}

From this, we can see that $a(y) * y = (a_2, a_1, a_0, a_3)$ in vector form. When we compare the vector form of $a(y)*y$, $(a_2, a_1, a_0, a_3)$ to the vector form of $a(y)$, $(a_3, a_2, a_1, a_0)$, we can see that the bytes of $a(y)*y$ are the same bytes of $a(y)$, except that they have been circularly shifted to the left by one. Therefore, in AES MIXCOLUMNS arithmetic, the multiplication of any 4-byte vector $a$ will result in its bytes being shifted circularly left by one byte.
\\

\item For this question, we are asked to prove that in AES MIXCOLUMNS arithmetic, $y^i = y^j$ for any integer $i \geq 0$ where $j \equiv i$  $(mod 4)$ with $0 \leq j \leq 3$. If $i \equiv j$ mod (4), and $i$ is an integer, then $i$ can be rewritten in the form $i = 4k + j$, where $k$ is an integer such that $i/4 = k$. Using this assertion, we can turn the equation $y^i = y^j$ into:

\begin{align*}
    y^i &= y^j \\
    y^{4k + j} &= y^j \\
    (y^4)^k y^j &= y^j.
\end{align*}

We are told that in AES MIXCOLUMNS arithmetic, $M(y) = y^4 + 1 = 0$. From this, we find that $y^4 = 1$. We can use this equation substitute $y^4$ with $1$, giving us:

\begin{align*}
    (1)^k y^j &= y^j \\
    y^j &= y^j.
\end{align*}

Thus, we can see that in this arithmetic, $y^i = y^j$ for any integer $i \geq 0$ where $j \equiv i$ (mod 4) with $0 \leq j \leq 3$.
\\

\item We are asked to prove that in this arithmetic, the multiplication of any 4-byte vector by $y^i \geq 0$ is a circular left shift by $j$ bytes, where $j \equiv i$ (mod 4) with $0 \leq j \leq 3$. Suppose that $a$ is a 4-byte vector represented as $(a_3, a_2, a_1, a_0)$ .Let $a(y) = a_3y^3 + a_2y^2 + a_1y + a_0$ be the polynomial representation of \textit{a}. In this arithmetic, we are told that $M(y) = y^4 + 1$ and that $M(y) = 0$. From this we get $y^4 = 1$. From part (a)(ii) we know that $y^i = y^j$ for any integer $i \geq 0$ where $j \equiv i$ (mod 4) with $0 \leq j \leq 3$. Therefore, there are four cases to be examined: \\

\textbf{Case 1: j = 0.}
If $j = 0$, then $y^i = y^j = y^0 = 1$. Therefore, when we multiply $a(y)$ with $y^i$, we get:

\begin{align*}
    a(y) * y^i &= a(y) * 1 \\
    &= a(y).
\end{align*}

Therefore, $a(y) * y^0 = a(y) = (a_3, a_2, a_1, a_0)$, which is a left  circular shift of $a$ by 0 bytes. Thus, in this case the statement is proven true.
\\

\textbf{Case 2: j = 1.}
If $j = 1$, then $y^i = y^j = y^1 = y$. Therefore, when we multiply $a(y)$ with $y^i$, we get:

\begin{align*}
    a(y) * y^i &= a(y) * y \\
    &= (a_3y^3 + a_2y^2 + a_1y + a_0) * y \\
    &= a_3y^4 + a_2y^3 + a_1y^2 + a_0y.
\end{align*}

Using the fact that in this arithmetic $y^4 = 1$, we can reduce this equation to:

\begin{align*}
    a(y) * y &= a_3 + a_2y^3 + a_1y^2 + a_0y
    &= a_2y^3 + a_1y^2 + a_0y + a_3.
\end{align*}

Therefore, $a(y) * y^1 = a(y) = (a_2, a_1, a_0, a_3)$, which is a left  circular shift of $a$ by 1 byte. Thus, in this case the statement is proven true.
\\

\textbf{Case 3: j = 2.}
If $j = 2$, then $y^i = y^j = y^2$. Therefore, when we multiply $a(y)$ with $y^i$, we get:

\begin{align*}
    a(y) * y^i &= a(y) * y^2 \\
    &= (a_3y^3 + a_2y^2 + a_1y + a_0) * y^2 \\
    &= a_3y^5 + a_2y^4 + a_1y^3 + a_0y^2.
\end{align*}

Using the fact that in this arithmetic $y^4 = 1$, we can reduce this equation to:

\begin{align*}
    a(y) * y &= a_3y + a_2 + a_1y^3 + a_0y^2 \\
    &= a_1y^3 + a_0y^2 + a_3y + a_2.
\end{align*}

Therefore, $a(y) * y^2 = a(y) = (a_1, a_0, a_3, a_2)$, which is a left  circular shift of $a$ by 2 bytes. Thus, in this case the statement is proven true.
\\

\textbf{Case 4: j = 3.}
If $j = 3$, then $y^i = y^j = y^3$. Therefore, when we multiply $a(y)$ with $y^i$, we get:

\begin{align*}
    a(y) * y^i &= a(y) * y^3 \\
    &= (a_3y^3 + a_2y^2 + a_1y + a_0) * y^3 \\
    &= a_3y^6 + a_2y^5 + a_1y^4 + a_0y^3.
\end{align*}

Using the fact that in this arithmetic $y^4 = 1$, we can reduce this equation to:

\begin{align*}
    a(y) * y &= a_3y^2 + a_2y + a_1 + a_0y^3 \\
    &= a_0y^3 + a_3y^2 + a_2y + a_1.
\end{align*}

Therefore, $a(y) * y^3 = a(y) = (a_0, a_3, a_2, a_1)$, which is a left  circular shift of $a$ by 3 bytes. Thus, in this case the statement is proven true.
\\

From this, we can see that the statement holds for all cases. Therefore, the statement is true. 

\end{enumerate}

\item \begin{enumerate}
\item In the Rijndahl field GF($2^8)$, the bytes (01), (02), and (03) are, respectively:
\begin{align*}
    c_1(x) = 1\\
    c_2(x) = x\\
    c_3(x) = x+1.
\end{align*} 
\\

\item From the previous part, we know that the Rijndahl representation of (02) is $c_2(x) = x$. The representation of \textit{b} in the Rijndahl field GF($2^8$), $b(x)$, is:

\begin{align*}
    b(x) = b_7x^7 + b_6x^6 + b_5x^5 + b_4x^4 + b_3x^3 + b_2x^2 + b_1x^1 + b_0.
\end{align*}

Therefore, the value of $d = (02)b$ in the Rijndahl field can be computed as:

\begin{align*}
    d &= (02)b \\
    d(x) &= c_2(x)b(x) \\
    &= (x)(b_7x^7 + b_6x^6 + b_5x^5 + b_4x^4 + b_3x^3 + b_2x^2 + b_1x^1 + b_0) \\
    &= b_7x^8 + b_6x^7 + b_5x^6 + b_4x^5 + b_3x^4 + b_2x^3 + b_1x^2 + b_0x.
\end{align*}

We are told that in this field, arithmetic is done modulo $m(x)$, where $m(x) = x^8 + x^4 + x^3 + x + 1$. We can use the fact that the modulus for a given modular arithmetic is always zero for the corresponding modular arithmetic to determine that $m(x) = 0$, since $m(x)$ is the modulus that corresponds to the Rijndahl field GF($2^8$). Since $m(x) = 0 = x^8 + x^4 + x^3 + x + 1$, we find that $x^8 = x^4 + x^3 + x + 1$ in this field. We can substitute this into the expression determined for $d = (02)b$ to obtain:

\begin{align*}
    d(x) &= b_7x^8 + b_6x^7 + b_5x^6 + b_4x^5 + b_3x^4 + b_2x^3 + b_1x^2 + b_0x \\
    &= b_7(x^4 + x^3 + x + 1) + b_6x^7 + b_5x^6 + b_4x^5 + b_3x^4 + b_2x^3 + b_1x^2 + b_0x \\
    d(x) &= b_6x^7 + b_5x^6 + b_4x^5 + (b_3 + b_7)x^4 + (b_2 + b_7)x^3 + b_1x^2 + (b_0 + b_7)x + b7.
\end{align*}

Thus, we have determine the expression for $d(x)$. Given that $d$ is a byte in the form $d = (d_7d_6d_5 \ldots d_1d0)$, we can write the symbolic expression for each bit $d_i$ of $d$ in terms of the bits of $b$:

\begin{align*}
    d_7 &= b_6 \\
    d_6 &= b_5 \\
    d_5 &= b_4 \\
    d_4 &= b_3 + b_7 \\
    d_3 &= b_2 + b_7 \\
    d_2 &= b_1 \\
    d_1 &= b_0 + b_7 \\
    d_0 &= b_7.
\end{align*}

\item From the part (i), we know that the Rijndahl representation of (03) is $c_3(x) = x + 1$. The representation of \textit{b} in the Rijndahl field GF($2^8$), $b(x)$, is:

\begin{align*}
    b(x) = b_7x^7 + b_6x^6 + b_5x^5 + b_4x^4 + b_3x^3 + b_2x^2 + b_1x^1 + b_0.
\end{align*}

Therefore, the value of $e = (03)b$ in the Rijndahl field can be computed as:

\begin{align*}
    e &= (03)b \\
    e(x) &= c_3(x)b(x) \\
    &= (x + 1)(b_7x^7 + b_6x^6 + b_5x^5 + b_4x^4 + b_3x^3 + b_2x^2 + b_1x^1 + b_0) \\
    &= b_7x^8 + b_6x^7 + b_5x^6 + b_4x^5 + b_3x^4 + b_2x^3 + b_1x^2 + b_0x +\\
    & \quad b_7x^7 + b_6x^6 + b_5x^5 + b_4x^4 + b_3x^3 + b_2x^2 + b_1x^1 + b_0 \\
    &= b_7x^8 + (b_6 + b_7)x^7 + (b_5 + b_6)x^6 + (b_4 + b_5)x^5 + (b_3 + b_4)x^4 + (b_2 + b_3)x^3 +\\
    & \quad (b_1 + b_2)x^2 + (b_0 + b_1)x^1 + b_0.
\end{align*}

We are told that in this field, arithmetic is done modulo $m(x)$, where $m(x) = x^8 + x^4 + x^3 + x + 1$. We can use the fact that the modulus for a given modular arithmetic is always zero for the corresponding modular arithmetic to determine that $m(x) = 0$, since $m(x)$ is the modulus that corresponds to the Rijndahl field GF($2^8$). Since $m(x) = 0 = x^8 + x^4 + x^3 + x + 1$, we find that $x^8 = x^4 + x^3 + x + 1$ in this field. We can substitute this into the expression determined for $e = (03)b$ to obtain:

\begin{align*}
    e(x) &= b_7x^8 + (b_6 + b_7)x^7 + (b_5 + b_6)x^6 + (b_4 + b_5)x^5 + (b_3 + b_4)x^4 + (b_2 + b_3)x^3 + \\
    & \quad (b_1 + b_2)x^2 + (b_0 + b_1)x^1 + b_0 \\
    &= b_7(x^4 + x^3 + x + 1) + (b_6 + b_7)x^7 + (b_5 + b_6)x^6 + (b_4 + b_5)x^5 + (b_3 + b_4)x^4 + \\
    & \quad (b_2 + b_3)x^3 + (b_1 + b_2)x^2 + (b_0 + b_1)x^1 + b_0 \\
    e(x) &= (b_6 + b_7)x^7 + (b_5 + b_6)x^6 + (b_4 + b_5)x^5 + (b_3 + b_4 + b_7)x^4 + (b_2 + b_3 + b_7)x^3 + \\ & \quad (b_1 + b_2)x^2 + (b_0 + b_1 + b_7)x^1 + (b_0 + b_7).
\end{align*}

Thus, we have determine the expression for $e(x)$. Given that $e$ is a byte in the form $e = (e_7e_6e_5 \ldots e_1e0)$, we can write the symbolic expression for each bit $e_i$ of $e$ in terms of the bits of $b$:

\begin{align*}
    e_7 &= b_6 + b_7 \\
    e_6 &= b_5 + b_6 \\
    e_5 &= b_4 + b_5 \\
    e_4 &= b_3 + b_4 + b_7 \\
    e_3 &= b_2 + b_3 + b_7 \\
    e_2 &= b_1 + b_2 \\
    e_1 &= b_0 + b_1 + b_7 \\
    e_0 &= b_0 + b_7.
\end{align*}
\end{enumerate}

\item

\begin{enumerate}
\item From part (b), we know that $c(y) = (03)y^3 + (01)y^2 + (01)y + (02)$. Therefore, given that $s(y) = s_3y^3 + s_2y^2 + s_1y + s_0$, we can compute $t(y) = s(y)c(y)$ as:

\begin{align*}
    t(y) &= s(y)c(y) \\
    &= (s_3y^3 + s_2y^2 + s_1y + s_0)((03)y^3 + (01)y^2 + (01)y + (02)) \\ \\
    &= (03)s_3y^6 + (03)s_2y^5 + (03)s_1y^4 + (02)s_0y^3 + \\
    & \quad (01)s_3y^5 + (01)s_2y^4 + (01)s_1y^3 + (01)s_0y^2 + \\
    & \quad (01)s_3y^4 + (01)s_2y^3 + (01)s_1y^2 + (01)s_0y + \\
    & \quad (02)s_3y^3 + (02)s_2y^2 + (02)s_1y + (02)s_0.
\end{align*}

From the previous parts, we know that in this arithmetic, $M(y) = y^4 + 1 = 0$, meaning that $y^4 = 1$. Using this fact, we can reduce the above expression for $t(y)$:

\begin{align*}
    t(y) &= s(y)c(y) \\
    &= (s_3y^3 + s_2y^2 + s_1y + s_0)((03)y^3 + (01)y^2 + (01)y + (02)) \\ \\
    &= (03)s_3y^6 + (03)s_2y^5 + (03)s_1y^4 + (02)s_0y^3 + \\
    & \quad (01)s_3y^5 + (01)s_2y^4 + (01)s_1y^3 + (01)s_0y^2 + \\
    & \quad (01)s_3y^4 + (01)s_2y^3 + (01)s_1y^2 + (01)s_0y + \\
    & \quad (02)s_3y^3 + (02)s_2y^2 + (02)s_1y + (02)s_0 \\ \\
    &= (03)s_3y^2 + (03)s_2y + (03)s_1 + (02)s_0y^3 + \\
    & \quad (01)s_3y + (01)s_2 + (01)s_1y^3 + (01)s_0y^2 + \\
    & \quad (01)s_3 + (01)s_2y^3 + (01)s_1y^2 + (01)s_0y + \\
    & \quad (02)s_3y^3 + (02)s_2y^2 + (02)s_1y + (02)s_0 \\ \\
    &= [(03)s_0 + (01)s_1 + (01)s_2 + (02)s_3]y^3 + \\
    & \quad [(01)s_0 + (01)s_1 + (02)s_2 + (03)s_3]y^2 + \\
    & \quad [(01)s_0 + (02)s_1 + (03)s_2 + (01)s_3]y + \\
    & \quad [(02)s_0 + (03)s_1 + (01)s_2 + (01)s_3]
\end{align*}

The polynomial $t(y) = t_3y^3 + t_2y^2 + t_1y + t_0$ can be written has a vector $(t_0, t_1, t_2, t_3)$. The symbolic expression for each byte in this vector, in terms of the bytes of $s(y)$, can be written as:

\begin{align*}
    t_0 &= (02)s_0 + (03)s_1 + (01)s_2 + (01)s_3 \\
    t_1 &= (01)s_0 + (02)s_1 + (03)s_2 + (01)s_3 \\
    t_2 &= (01)s_0 + (01)s_1 + (02)s_2 + (03)s_3 \\
    t_3 &= (03)s_0 + (01)s_1 + (01)s_2 + (02)s_3.
\end{align*}

\item Using the symbolic expressions of the vector $(t_0, t_1, t_2, t_3)$ from the previous part, we can derive the matrix $C$ such that:

\begin{align*}
    \begin{pmatrix}
        t_0 \\
        t_1 \\
        t_2 \\
        t_3
    \end{pmatrix}
    = C
    \begin{pmatrix}
        s_0 \\
        s_1 \\
        s_2 \\
        s_3
    \end{pmatrix}.
\end{align*}

The matrix C is:

\begin{align*}
    C = \begin{bmatrix}
    (02) & (03) & (01) & (01) \\
    (01) & (02) & (03) & (01) \\
    (01) & (01) & (02) & (03) \\
    (03) & (01) & (01) & (02)
  \end{bmatrix}
\end{align*}.

\end{enumerate}
\end{enumerate}

\newpage

\stepcounter{problem}
\item[] \textbf{Problem \theproblem} --- Error propagation in block cipher modes (12 marks)

\begin{enumerate}
\item

\begin{enumerate}
\item In ECB, each block of the ciphertext is decrypted individually ($D_K(C_i) = M_i$). The ciphertext blocks have no bearing on the key either; they're completely independent. Therefore, an error in $C_i$ will only affect the plaintext block $M_i$.
\\
\item In CBC, each block of the ciphertext is decrypted using the both the current ciphetext block and the previous ciphertext block ($M_i = D_K(C_i) \oplus C_{i-1}$). Therefore, when $C_i$ is being decrypted, its error is propagated to $M_i$, since $M_i = D_K(C_i) \oplus C_{i-1}$ and $C_i$ has an error. In addition to this, the error in $C_i$ is also propagated to message block $M_{i+1}$, since $M_{i+1} = D_K(C_{i+1}) \oplus C_{i}$. Therefore, an error in ciphertext block $C_i$ propagates to message blocks $M_i$ and $M_{i+1}$ in CBC.
\\
\item In OFB, each block of the ciphertext is decrypted by XORing the ciphertext block with a pseudorandom key stream ($M_i = C_i \oplus KS_i$). The initial key stream ($KS_0$) is assigned a random initial value. Every subsequent key steam $KS_i$ is generated through the equation $KS_i = E_K(KS_{i-1})$. The ciphertext blocks themselves are independent from the generation of the key stream values. Therefore, an error in $C_i$ will only affect message block $M_i$.
\\
\item In CFB with one register, each block of the ciphertext is decrypted by XORing the ciphertext block with a pseudorandom key stream ($M_i = C_i \oplus KS_i$). The initial key stream $KS_0$ is generated by encrypting some random initial value IV ($KS_0 = E_K(IV)$). Every subsequent key stream $KS_i$ is generated using the ciphertext block from the previous decryption ($KS_i = E_K(C_{i-1})$). An error in $C_i$ will result in an error in the decryption of $M_i$, since it depends on the value of $C_i$ ($M_i = C_i \oplus KS_i$). Furthermore, an error in $C_i$ will result in an error in the generation of $KS_{i+1}$, since $KS_{i+1} = E_K(C_{i})$. Since $KS_{i+1}$ has an error, this means that the decryption of message block $M_{i+1}$ will also contain an error, since $M_{i+1} = C_{i+1} \oplus KS_{i+1}$. Therefore, an error in ciphertext block $C_i$ will result in errors in the decryptions of both message block $M_i$ and message block $M_{i+1}$.
\\
\item In CTR, each block of the ciphertext is decrypted by XORing the ciphertext block with a pseudorandom key stream ($M_i = C_i \oplus KS_i$). Each value of $KS_i$ is generate by encrypting a counter value $CTR_i$ of the same size as the plaintext block size ($KS_i = E_K(CTR_i)$). Each subsequent counter value $CTR_{i+1}$ is generated using the previous counter value $CTR_i$. This means that the generation of key streams is independent from the ciphertext blocks. Therefore, an error in $C_i$ will only result in an error in message block $M_i$, since $M_i = C_i \oplus KS_i$.
\\
\end{enumerate}

\item In CBC, each block of the plaintext is encrypted using both the current message block $M_i$ and the last generated ciphertext block $C_{i-1}$, or the initial value IV ($C_0$). If there is an error in message block $M_i$, this will propagate to ciphertext $C_i$, since $C_i = E_K(M_i \oplus C_{i-1})$. Furthermore, since $C_i$ has errors, this results in $C_{i+1}$ also having errors, since $C_{i+1} = E_K(M_i \oplus C_i)$. The error in $C_{i+1}$ also propagates to $C_{i+2}$, for the same reason. Therefore, if there is an error in $M_i$ during encryption, this error will propagate into $C_i, C_{i+1}, \ldots, C_n$, where $C_n$ is the final ciphertext block generated during encryption.

Since there's an error in $C_i$, that means that during decryption, there will be an error in message block $M_i$, since during decryption ($M_i = D_K(C_i) \oplus C_{i-1}$). However, for the remaining message blocks $M_{i+1}, M_{i+2}, \ldots, M_n$, there is no error during decryption of these message blocks. The decryption of $M_{i+1}$, involves the equation $M_{i+1} = D_K(C_{i+1}) \oplus C_i$. However, from the encryption process, we know that $C_{i+1} = E_K(M_{i+1} \oplus C_i)$. If we substitute this into the decryption equation, we see that: 
\begin{align*}
    M_{i+1} &= D_K(C_{i+1}) \oplus C_i \\
    &= D_K(E_K(M_{i+1} \oplus C_i)) \oplus C_i \\
    &= M_{i+1} \oplus C_i \oplus C_i \\
    &= M_{i+1}.
\end{align*}

From this, we can see that in spite of the ciphertext block having an error introduced during the encryption phase, it doesn't matter during the decryption, since the error that's been introduced is removed when the ciphertext block is XOR'd against itself (since any value XOR'd with itself equals 0). The only reason that $M_i$ contains an error after decryption is because the original message block $M_i$ during encryption also contained an error. Since the error for this message block originated in the plaintext, it remains after decryption. However, for every subsequent block in the encryption process, the error was only introduced during the generation of the ciphertext block, so the error doesn't persist upon decryption.

\end{enumerate}

\newpage


\stepcounter{problem}
\item[] \textbf{Problem \theproblem} ---  Binary exponentiation (13 marks)

\begin{enumerate}
\item For this part, we're asked to perform the exponentiation algorithm, which is applied to evaluate expressions of the form $a^n$ (mod m), to evaluate $17^{11}$ mod (77).\\

First, we must compute the binary representation of 11. We're trying to represent 11 in the form $11 = b_02^k + b_12^{k-1} + \ldots + b_{k-1}2 + b_k$, where $b_0 = 1$ and $b_i \in {0, 1}$ for $1 \leq i \leq k$, and $k = \lfloor log_2n \rfloor$. $k = \lfloor log_2(11) \rfloor = 3$. 11 can be represented as $(1)2^3 + (0)2^2 + (1)2 + (1)1$. Therefore, $b_0 = 1, b_1 = 0, b_2 = 1, b_3 = 1$.\\

Next, we initialize the value $r_0$ such that $r_0 \equiv a$ mod (m). In this case, $r_0 = 17$ (mod 77).\\

Next, for a given $r_i$, we can compute $r_{i+1}$ using the following rules:\\

\textbf{If $b_{i+1} = 0$, then $r_{i+1} = r_i^2$} (mod m) \\ \\
\textbf{If $b_{i+1} = 1$, then $r_{i+1} = r_i^2 a$} (mod m) \\ \\

This process is repeated for $0 \leq i \leq k - 1$. Since $k = 3$ in this case, that means this process is repeated for $0 \leq i \leq 2$. The process is as follows: \\

For i = 0, $b_{i+1} = b_{0+1} = b{1} = 0$. Therefore, since $b_1 = 0$:

\begin{align*}
    r_{0+1} &= r_1 \\
    &= r_0^2 \mod{77} \\
    r_1 &= 17^2 \mod{77} \\
    &= 289 \mod{77} \\
    r_1 &= 58.
\end{align*}

For i = 1, $b_{i+1} = b_{1+1} = b{2} = 1$. Therefore, since $b_2 = 1$:

\begin{align*}
    r_{1+1} &= r_2 \\
    &= r_1^2a \mod{77} \\
    r_2 &= 58^2*17 \mod{77} \\
    &= 57188 \mod{77} \\
    r_2 &= 54.
\end{align*}

For i = 2, $b_{i+1} = b_{2+1} = b{3} = 1$. Therefore, since $b_3 = 1$:

\begin{align*}
    r_{2+1} &= r_3 \\
    &= r_2^2a \mod{77} \\
    r_3 &= 54^2*17 \mod{77} \\ 
    &= 49572 \mod{77} \\
    r_3 &= 61.
\end{align*}

Therefore, at the end of our approach, we find that $r_3 = 61$. Therefore, this means that $17^{11} \mod{77} = 61$.

\item
\begin{enumerate}
\item In this question, we are asked to prove that $s_i = \sum_{j=0}^i b_j2^{i-j}$ for $0 \leq i \leq k$, given that $s_0 = 1$ and $s_{i+1} = 2s_i + b_{i+1}$ for $0 \leq i \leq k - 1$. We shall prove this using induction. \\

\textbf{Base Case i = 0:} If $i = 0$, then $s_0 = 1$, as defined in the problem.\\

\textbf{Inductive Hypothesis:} Suppose that $s_i = \sum_{j=0}^i b_j2^{i-j}$ for $0 \leq i \leq k - 1$. \\

\textbf{Inductive Step:} We want to show that this inductive hypothesis also holds for $i+1$. From the question, we are provided with the definition that $s_{i+1} = 2s_i + b_{i+1}$ for $0 \leq i \leq k$. From the inductive hypothesis we know that $s_i = \sum_{j=0}^i b_j2^{i-j}$. We can substitute this into the previous equation to get: $s_{i+1} = 2(\sum_{j=0}^i b_j2^{i-j}) + b_{i+1}$. The $2$ multiplying the summation can be moved into the equation to get the following: $s_{i+1} = \sum_{j=0}^i b_j2^{i-j}2 + b_{i+1}$. Using power rules, we can re-write this into $s_{i+1} = \sum_{j=0}^i b_j2^{i+1-j} + b_{i+1}$. The term $b_{i+1}$ can be re-written as $b_{i+1}2^0 = b_{i+1}2^{i+1-(i+1)}$. This given us:

\begin{align*}
    s_{i+1} = \sum_{j=0}^i b_j2^{i+1-j} + b_{i+1}2^{i+1-(i+1)}
\end{align*}

The second term is of the form $b_j2^{i+1-j}$ where $j = i + 1$. This means that it can be added to the summation by increasing the upper limit on the summation from $i$ to $i + 1$. Finally, we see that:

\begin{align*}
    s_{i+1} = \sum_{j=0}^{i+1} b_j2^{i+1-j}.
\end{align*}

Thus, we have proven that the inductive hypothesis applies to $i+1$.

\item In this question, we are asked to prove that $r_0 \equiv a \mod{m}$ for $0 \leq i \leq k$, given the definitions provided in steps 2 and 3 of the exponentiation algorithm. We will prove this using induction. \\

\textbf{Base Case i = 0:} If $i = 0$, then we get $r_0 \equiv a^{s_0} \mod{m}$. From the previous part, we know that $s_0 = 1$. Therefore, this equation becomes: $r_0 \equiv a \mod{m}$. This is identical to the definition provided in step 2 of the exponentiation algorithm. Therefore, the base case is proven to be verified. \\

\textbf{Inductive Hypothesis:} Suppose that $r_i \equiv a^{s_i}$ for $0 \leq i \leq k - 1$. \\

\textbf{Inductive Step:} We want to prove that $r_{i+1} \equiv a^{s_{i+1}} \mod{m}$. We can compute $s_{i+1}$ using the definition provided in the previous part: $s_{i+1} = 2s_i + b_{i+1}$. Since the value of $b_{i+1}$ is unknown, there are two cases that must be considered. \\

\textbf{Case 1, $b_{i+1} = 0$}: If $b_{i+1} = 0$, then $s_{i+1} = 2s_i + 0 = 2s_i$. Therefore, $r_{i+1} \equiv a^{2s_i} \mod{m} \equiv (a^{s_i})^2 \mod{m}$. From the inductive hypothesis, we know that $r_i \equiv a^{s_i} \mod{m}$. Therefore, when $b_{i+1} = 0$, then $r_{i+1} \equiv r_i^2 \mod{m}$. This is consistent with what is show in step 3 of the exponentiation algorithm when $b_{i+1} = 0$. Thus, we have proven that $r_{i+1} \equiv a^{s_{i+1}}$ in this case. \\

\textbf{Case 2, $b_{i+1} = 1$}: If $b_{i+1} = 1$, then $s_{i+1} = 2s_i$. Therefore, $r_{i+1} \equiv a^{2s_i+1} \mod{m} \equiv a^{2s_i}a \mod{m} \equiv (a^{s_i})^2a \mod{m}$. From the inductive hypothesis, we know that $r_i \equiv a^{s_i} \mod{m}$. Therefore, when $b_{i+1} = 1$, then $r_{i+1} \equiv r_i^2a \mod{m}$. This is consistent with what is show in step 3 of the exponentiation algorithm when $b_{i+1} = 1$. Thus, we have proven that $r_{i+1} \equiv a^{s_{i+1}}$ in this case. \\

Since we have proven that $r_{i+1} \equiv a^{s_{i+1}}$, in both cases, the statement holds in all cases. Therefore, we have proven that the inductive hypothesis applies to $i+1$.

\item We are asked to prove that $a^n \equiv r_k$. In part (i), we proved that $s_i = \sum_{j=0}^{i} b_j2^{i-j}$ for $0 \leq i \leq k$. If we set $i = k$, we find that $s_k = \sum_{j=0}^{k} b_j2^{i-j}$. When we expand this, we find that:

\begin{align*}
    s_k &= b_02^{k-0} + b_12^{k-1} + \ldots + b_{k-1}2^{k-(k-1)} + b_k2^{k-k} \\
    &= b_02^k + b_12^{k-1} + \ldots + b_{k-1}2 + b_k.
\end{align*}

This expression for $s_k$ perfectly matches step one of the exponentiation algorithm, where the exponent $n$ in the expression $a^n \mod{m}$ is written in the form: $n = b_02^k + b_12^{k-1} + \ldots + b_{k-1}2 + b_k$. From this, we can conclude that $s_k = n$. \\

Next, in part (ii), we proved that $r_i \equiv a^{s_i}$ for $0 \leq i \leq k$. If we set $i = k$, we can see that $r_k = a^{s_k} \mod{m}$. Earlier, we found that $s_k = n$. If we substitute this equality in, we see that $r_k \equiv a^n \mod{m}$. This can be re-written as $a^n \equiv r_k \mod{m}$, since modulo arithmetic is symmetrical. Thus, we have proven that $a^n \equiv r_k \mod{m}$, thereby verifying that the exponentiation algorithm does compute $a^n \mod{m}$ like it claims.

\end{enumerate}
\end{enumerate}

\newpage

\stepcounter{problem}
\item[] \textbf{Problem \theproblem} --- A modified man-in-the-middle attack on Diffie-Hellman
    (10 marks)

\begin{enumerate}
\item Mallory sends Alice $(g^b)^q \mod{m}$. She then computes $Key_{Alice}$ as $((g^b)^q)^a \mod{m} = g^{abq} \mod{m}$. Therefore, $Key_{Alice} = g^{abq}$. \\

Mallory sends Bob $(g^a)^q \mod{m}$. He then computes $Key_{Bob}$ as $((g^a)^q)^b \mod{m} = g^{abq} \mod{m}$. Therefore, $Key_{Bob} = g^{abq}$. \\

From this, we can see that $Key_{Alice} = Key_{Bob}$. Therefore, both Alice and Bob end up calculating the same shared key.

\item 

\item In the man-in-the-middle attack discussed in lecture, there are two shared keys: $g^{ea} \mod{m}$, which is shared between Alice and Mallory, and then there is $g^{eb} \mod{m}$, which is shared between Mallory and Bob. In this scenario, Mallory has to intercept any messages sent by Alice, decrypt them using the key shares with Alice, and then re-encrypt the message with the key she shares with Bob. If she fails to do this a single message, then Alice and Bob may notice that there is an interceptor. The reason is because Alice and Bob don't have a shared key with each other. This means that if Alice gets a message directly to Bob without Mallory intercepting it, Bob won't be able to decrypt it. Therefore, this may indicate to them that something suspicious is happening. This approach requires much more effort from Mallory, since she can't afford to miss a single message, lest she be found out. \\

In comparison, the attack explored in this question is far more passive. Mallory doesn't need to worry about intercepting every message, because Alice and Bob \textit{do} share a key in this case. Therefore, even if Mallory misses intercepting a message, the recipient can still decrypt it successfully. This greatly reduces the chances of her being caught.

\end{enumerate}

\newpage

\stepcounter{problem}
\item[] \textbf{Problem \theproblem} --- A simplified password-based key agreement protocol (8
    marks)

\begin{enumerate}
\item By the end of this procedure, $K_{Client} \equiv B^{a+p} \mod{N}$. However, we know from step 1 of the protocol that $B \equiv g^b \mod{N}$. Therefore, we can re-write $K_{Client}$ as $(g^b)^{a+p} \equiv \mod{N}$. This can be expanded out to obtain: $K_{Client} \equiv g^{ab + bp} \mod{N}$. \\

By the end of this procedure, $K_{Server} \equiv (Av)^{b} \mod{N}$. However, we know from the question's description of the protocol that $v \equiv g^p \mod{N}$. We also know from step one that $A \equiv g^a \mod{N}$. Therefore, we can re-write $K_{Server}$ as $(g^ag^p)^b \equiv \mod{N}$. This can be expanded out to obtain: $K_{Client} \equiv (g^{a + p})^b \mod{N} \equiv g^{ab + bp} \mod{N}$. Therefore, $K_{Server} = g^{ab + bp} \mod{N}$. \\

From this, we can see that $K_{Client} \equiv g^{ab + bp} \mod{N}$, and that $K_{Server} \equiv g^{ab + bp} \mod{N}$. Therefore, $K_{Client} = K_{Server}$.

\item 

\item
\end{enumerate}


\newpage

\stepcounter{problem}
\item[] \textbf{Problem \theproblem} --- Primitive roots for safe primes (6 marks)

\newpage

\stepcounter{problem}
\item[] \textbf{Problem \theproblem} --- Discrete logarithms with respect to different primitive
    roots (8 marks)

\newpage

\stepcounter{problem}
\item[] \textbf{Problem \theproblem} --- An algorithm for extracting discrete logarithms (21
    marks)

\begin{enumerate}
\item

\item

\item

\item

\item
    \begin{enumerate}
    \item

    \item
\end{enumerate}
\end{enumerate}

\newpage

\stepcounter{problem} \stepcounter{problem}

\item[] \textbf{Problem \theproblem} --- Playfair cipher cryptanalysis, 10 marks

\end{enumerate}

\end{document}
