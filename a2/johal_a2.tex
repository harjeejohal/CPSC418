\documentclass[11pt]{article}

\usepackage{amsthm,amsmath,amssymb,amsfonts}
\usepackage[margin=1in]{geometry}

\parindent 0pt
\parskip 3mm

\theoremstyle{definition}
\newtheorem*{solution}{Solution}
\newcounter{problem}

\begin{document}

\begin{center}
{\bf \Large CPSC 418 / MATH 318 --- Introduction to Cryptography

ASSIGNMENT 2 }
\end{center}

\hrule 	

\textbf{Name:} Harjee Johal \\
\textbf{Student ID:} 30000668 \\

\medskip \hrule

\begin{enumerate} \itemsep 20pt

\stepcounter{problem}
\item[] \textbf{Problem \theproblem} ---  Arithmetic in the AES {\sc MixColumns} operation (22
    marks)

\begin{enumerate}
\item \begin{enumerate}

\item (a) (i) For this question, we are asked to prove that in AES MIXCOLUMNS arithmetic multiplying any 4-byte vector by \textit{y} is a circular left shift by one byte. Suppose that \textit{a} is a 4-byte vector such that \textit{a = $(a_3, a_2, a_1, a_0)$}. Let $a(y) = a_3y^3 + a_2y^2 + a_1y + a_0$ be the polynomial representation of \textit{a}. Thus, $a(y) * y$ is:

\begin{align*}
    a(y)*y &= (a_3y^3 + a_2y^2 + a_1y + a_0)*y \\
    &= a_3y^4 + a_2y^3 + a_1y^2 + a_0y.
\end{align*}

Next, we must perform reduction of $a(y)*y$ modulo \textit{M(y)}. We are told that $M(y) = 0$ and that $M(y) = y^4 + 1$. From this, can determine that since $y^4+1 = 0$, then $y^4 = 1$. Applying this to $a(y)*y$, we get:

\begin{align*}
    a(y) * y &= a_3y^4 + a_2y^3 + a_1y^2 + a_0y \\
    &= a_3(1) + a_2y^3 + a_1y^2 + a_0y \\
    &= a_2y^3 + a_1y^2 + a_0y + a_3.
\end{align*}

From this, we can see that $a(y) * y = (a_2, a_1, a_0, a_3)$ in vector form. When we compare the vector form of $a(y)*y$, $(a_2, a_1, a_0, a_3)$ to the vector form of $a(y)$, $(a_3, a_2, a_1, a_0)$, we can see that the bytes of $a(y)*y$ are the same bytes of $a(y)$, except that they have been circularly shifted to the left by one. Therefore, in AES MIXCOLUMNS arithmetic, the multiplication of any 4-byte vector $a$ will result in its bytes being shifted circularly left by one byte.
\\

\item For this question, we are asked to prove that in AES MIXCOLUMNS arithmetic, $y^i = y^j$ for any integer $i \geq 0$ where $j \equiv i$  $(mod 4)$ with $0 \leq j \leq 3$. If $i \equiv j$ mod (4), and $i$ is an integer, then $i$ can be rewritten in the form $i = 4k + j$, where $k$ is an integer such that $i/4 = k$. Using this assertion, we can turn the equation $y^i = y^j$ into:

\begin{align*}
    y^i &= y^j \\
    y^{4k + j} &= y^j \\
    (y^4)^k y^j &= y^j.
\end{align*}

We are told that in AES MIXCOLUMNS arithmetic, $M(y) = y^4 + 1 = 0$. From this, we find that $y^4 = 1$. We can use this equation substitute $y^4$ with $1$, giving us:

\begin{align*}
    (1)^k y^j &= y^j \\
    y^j &= y^j.
\end{align*}

Thus, we can see that in this arithmetic, $y^i = y^j$ for any integer $i \geq 0$ where $j \equiv i$ (mod 4) with $0 \leq j \leq 3$.
\\

\item We are asked to prove that in this arithmetic, the multiplication of any 4-byte vector by $y^i \geq 0$ is a circular left shift by $j$ bytes, where $j \equiv i$ (mod 4) with $0 \leq j \leq 3$. Suppose that $a$ is a 4-byte vector represented as $(a_3, a_2, a_1, a_0)$ .Let $a(y) = a_3y^3 + a_2y^2 + a_1y + a_0$ be the polynomial representation of \textit{a}. In this arithmetic, we are told that $M(y) = y^4 + 1$ and that $M(y) = 0$. From this we get $y^4 = 1$. From part (a)(ii) we know that $y^i = y^j$ for any integer $i \geq 0$ where $j \equiv i$ (mod 4) with $0 \leq j \leq 3$. Therefore, there are four cases to be examined: \\

\textbf{Case 1: j = 0.}
If $j = 0$, then $y^i = y^j = y^0 = 1$. Therefore, when we multiply $a(y)$ with $y^i$, we get:

\begin{align*}
    a(y) * y^i &= a(y) * 1 \\
    &= a(y).
\end{align*}

Therefore, $a(y) * y^0 = a(y) = (a_3, a_2, a_1, a_0)$, which is a left  circular shift of $a$ by 0 bytes. Thus, in this case the statement is proven true.
\\

\textbf{Case 2: j = 1.}
If $j = 1$, then $y^i = y^j = y^1 = y$. Therefore, when we multiply $a(y)$ with $y^i$, we get:

\begin{align*}
    a(y) * y^i &= a(y) * y \\
    &= (a_3y^3 + a_2y^2 + a_1y + a_0) * y \\
    &= a_3y^4 + a_2y^3 + a_1y^2 + a_0y.
\end{align*}

Using the fact that in this arithmetic $y^4 = 1$, we can reduce this equation to:

\begin{align*}
    a(y) * y &= a_3 + a_2y^3 + a_1y^2 + a_0y
    &= a_2y^3 + a_1y^2 + a_0y + a_3.
\end{align*}

Therefore, $a(y) * y^1 = a(y) = (a_2, a_1, a_0, a_3)$, which is a left  circular shift of $a$ by 1 byte. Thus, in this case the statement is proven true.
\\

\textbf{Case 3: j = 2.}
If $j = 2$, then $y^i = y^j = y^2$. Therefore, when we multiply $a(y)$ with $y^i$, we get:

\begin{align*}
    a(y) * y^i &= a(y) * y^2 \\
    &= (a_3y^3 + a_2y^2 + a_1y + a_0) * y^2 \\
    &= a_3y^5 + a_2y^4 + a_1y^3 + a_0y^2.
\end{align*}

Using the fact that in this arithmetic $y^4 = 1$, we can reduce this equation to:

\begin{align*}
    a(y) * y &= a_3y + a_2 + a_1y^3 + a_0y^2 \\
    &= a_1y^3 + a_0y^2 + a_3y + a_2.
\end{align*}

Therefore, $a(y) * y^2 = a(y) = (a_1, a_0, a_3, a_2)$, which is a left  circular shift of $a$ by 2 bytes. Thus, in this case the statement is proven true.
\\

\textbf{Case 4: j = 3.}
If $j = 3$, then $y^i = y^j = y^3$. Therefore, when we multiply $a(y)$ with $y^i$, we get:

\begin{align*}
    a(y) * y^i &= a(y) * y^3 \\
    &= (a_3y^3 + a_2y^2 + a_1y + a_0) * y^3 \\
    &= a_3y^6 + a_2y^5 + a_1y^4 + a_0y^3.
\end{align*}

Using the fact that in this arithmetic $y^4 = 1$, we can reduce this equation to:

\begin{align*}
    a(y) * y &= a_3y^2 + a_2y + a_1 + a_0y^3 \\
    &= a_0y^3 + a_3y^2 + a_2y + a_1.
\end{align*}

Therefore, $a(y) * y^3 = a(y) = (a_0, a_3, a_2, a_1)$, which is a left  circular shift of $a$ by 3 bytes. Thus, in this case the statement is proven true.
\\

From this, we can see that the statement holds for all cases. Therefore, the statement is true. 

\end{enumerate}

\item \begin{enumerate}
\item In the Rijndahl field GF($2^8)$, the bytes (01), (02), and (03) are, respectively:
\begin{align*}
    c_1(x) = 1\\
    c_2(x) = x\\
    c_3(x) = x+1.
\end{align*} 
\\

\item From the previous part, we know that the Rijndahl representation of (02) is $c_2(x) = x$. The representation of \textit{b} in the Rijndahl field GF($2^8$), $b(x)$, is:

\begin{align*}
    b(x) = b_7x^7 + b_6x^6 + b_5x^5 + b_4x^4 + b_3x^3 + b_2x^2 + b_1x^1 + b_0.
\end{align*}

Therefore, the value of $d = (02)b$ in the Rijndahl field can be computed as:

\begin{align*}
    d &= (02)b \\
    d(x) &= c_2(x)b(x) \\
    &= (x)(b_7x^7 + b_6x^6 + b_5x^5 + b_4x^4 + b_3x^3 + b_2x^2 + b_1x^1 + b_0) \\
    &= b_7x^8 + b_6x^7 + b_5x^6 + b_4x^5 + b_3x^4 + b_2x^3 + b_1x^2 + b_0x.
\end{align*}

We are told that in this field, arithmetic is done modulo $m(x)$, where $m(x) = x^8 + x^4 + x^3 + x + 1$. We can use the fact that the modulus for a given modular arithmetic is always zero for the corresponding modular arithmetic to determine that $m(x) = 0$, since $m(x)$ is the modulus that corresponds to the Rijndahl field GF($2^8$). Since $m(x) = 0 = x^8 + x^4 + x^3 + x + 1$, we find that $x^8 = x^4 + x^3 + x + 1$ in this field. We can substitute this into the expression determined for $d = (02)b$ to obtain:

\begin{align*}
    d(x) &= b_7x^8 + b_6x^7 + b_5x^6 + b_4x^5 + b_3x^4 + b_2x^3 + b_1x^2 + b_0x \\
    &= b_7(x^4 + x^3 + x + 1) + b_6x^7 + b_5x^6 + b_4x^5 + b_3x^4 + b_2x^3 + b_1x^2 + b_0x \\
    d(x) &= b_6x^7 + b_5x^6 + b_4x^5 + (b_3 + b_7)x^4 + (b_2 + b_7)x^3 + b_1x^2 + (b_0 + b_7)x + b7.
\end{align*}

Thus, we have determine the expression for $d(x)$. Given that $d$ is a byte in the form $d = (d_7d_6d_5 \ldots d_1d0)$, we can write the symbolic expression for each bit $d_i$ of $d$ in terms of the bits of $b$:

\begin{align*}
    d_7 &= b_6 \\
    d_6 &= b_5 \\
    d_5 &= b_4 \\
    d_4 &= b_3 + b_7 \\
    d_3 &= b_2 + b_7 \\
    d_2 &= b_1 \\
    d_1 &= b_0 + b_7 \\
    d_0 &= b_7.
\end{align*}

\item From the part (i), we know that the Rijndahl representation of (03) is $c_3(x) = x + 1$. The representation of \textit{b} in the Rijndahl field GF($2^8$), $b(x)$, is:

\begin{align*}
    b(x) = b_7x^7 + b_6x^6 + b_5x^5 + b_4x^4 + b_3x^3 + b_2x^2 + b_1x^1 + b_0.
\end{align*}

Therefore, the value of $e = (03)b$ in the Rijndahl field can be computed as:

\begin{align*}
    e &= (03)b \\
    e(x) &= c_3(x)b(x) \\
    &= (x + 1)(b_7x^7 + b_6x^6 + b_5x^5 + b_4x^4 + b_3x^3 + b_2x^2 + b_1x^1 + b_0) \\
    &= b_7x^8 + b_6x^7 + b_5x^6 + b_4x^5 + b_3x^4 + b_2x^3 + b_1x^2 + b_0x + b_7x^7 + b_6x^6 + b_5x^5 + b_4x^4 + b_3x^3 + b_2x^2 + b_1x^1 + b_0 \\
    &= b_7x^8 + (b_6 + b_7)x^7 + (b_5 + b_6)x^6 + (b_4 + b_5)x^5 + (b_3 + b_4)x^4 + (b_2 + b_3)x^3 + (b_1 + b_2)x^2 + (b_0 + b_1)x^1 + b_0.
\end{align*}

We are told that in this field, arithmetic is done modulo $m(x)$, where $m(x) = x^8 + x^4 + x^3 + x + 1$. We can use the fact that the modulus for a given modular arithmetic is always zero for the corresponding modular arithmetic to determine that $m(x) = 0$, since $m(x)$ is the modulus that corresponds to the Rijndahl field GF($2^8$). Since $m(x) = 0 = x^8 + x^4 + x^3 + x + 1$, we find that $x^8 = x^4 + x^3 + x + 1$ in this field. We can substitute this into the expression determined for $e = (03)b$ to obtain:

\begin{align*}
    e(x) &= b_7x^8 + (b_6 + b_7)x^7 + (b_5 + b_6)x^6 + (b_4 + b_5)x^5 + (b_3 + b_4)x^4 + (b_2 + b_3)x^3 + (b_1 + b_2)x^2 + (b_0 + b_1)x^1 + b_0 \\
    &= b_7(x^4 + x^3 + x + 1) + (b_6 + b_7)x^7 + (b_5 + b_6)x^6 + (b_4 + b_5)x^5 + (b_3 + b_4)x^4 + (b_2 + b_3)x^3 + (b_1 + b_2)x^2 + (b_0 + b_1)x^1 + b_0 \\
    e(x) &= (b_6 + b_7)x^7 + (b_5 + b_6)x^6 + (b_4 + b_5)x^5 + (b_3 + b_4 + b_7)x^4 + (b_2 + b_3 + b_7)x^3 + (b_1 + b_2)x^2 + (b_0 + b_1 + b_7)x^1 + (b_0 + b_7).
\end{align*}

Thus, we have determine the expression for $e(x)$. Given that $e$ is a byte in the form $e = (e_7e_6e_5 \ldots e_1e0)$, we can write the symbolic expression for each bit $e_i$ of $e$ in terms of the bits of $b$:

\begin{align*}
    e_7 &= b_6 + b_7 \\
    e_6 &= b_5 + b_6 \\
    e_5 &= b_4 + b_5 \\
    e_4 &= b_3 + b_4 + b_7 \\
    e_3 &= b_2 + b_3 + b_7 \\
    e_2 &= b_1 + b-2 \\
    e_1 &= b_0 + b_1 + b_7 \\
    e_0 &= b_0 + b_7.
\end{align*}
\end{enumerate}

\item

\begin{enumerate}
\item

\item
\end{enumerate}
\end{enumerate}

\newpage

\stepcounter{problem}
\item[] \textbf{Problem \theproblem} --- Error propagation in block cipher modes (12 marks)

\begin{enumerate}
\item

\begin{enumerate}
\item

\item

\item

\item

\item

\end{enumerate}

\item
\end{enumerate}

\newpage


\stepcounter{problem}
\item[] \textbf{Problem \theproblem} ---  Binary exponentiation (13 marks)

\begin{enumerate}
\item

\item
\begin{enumerate}
\item

\item

\item
\end{enumerate}
\end{enumerate}

\newpage

\stepcounter{problem}
\item[] \textbf{Problem \theproblem} --- A modified man-in-the-middle attack on Diffie-Hellman
    (10 marks)

\begin{enumerate}
\item

\item

\item

\end{enumerate}

\newpage

\stepcounter{problem}
\item[] \textbf{Problem \theproblem} --- A simplified password-based key agreement protocol (8
    marks)

\begin{enumerate}
\item

\item

\item
\end{enumerate}


\newpage

\stepcounter{problem}
\item[] \textbf{Problem \theproblem} --- Primitive roots for safe primes (6 marks)

\newpage

\stepcounter{problem}
\item[] \textbf{Problem \theproblem} --- Discrete logarithms with respect to different primitive
    roots (8 marks)

\newpage

\stepcounter{problem}
\item[] \textbf{Problem \theproblem} --- An algorithm for extracting discrete logarithms (21
    marks)

\begin{enumerate}
\item

\item

\item

\item

\item
    \begin{enumerate}
    \item

    \item
\end{enumerate}
\end{enumerate}

\newpage

\stepcounter{problem} \stepcounter{problem}

\item[] \textbf{Problem \theproblem} --- Playfair cipher cryptanalysis, 10 marks

\end{enumerate}

\end{document}
