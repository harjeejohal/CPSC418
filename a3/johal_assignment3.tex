\documentclass[11pt]{article}

\usepackage{amsthm,amsmath,amssymb,amsfonts}
\usepackage[margin=1in]{geometry}

\theoremstyle{definition}
\newtheorem*{solution}{Solution}
\newcounter{problem}

\newcommand{\CBCMAC}{\text{CBC-MAC}}
\newcommand{\AHMAC}{\mathrm{AHMAC}}
\newcommand{\PHMAC}{\mathrm{PHMAC}}
\renewcommand{\pmod}[1]{\mbox{\ $(\ensuremath{\operatorname{mod}}\ {#1})$}}
\newcommand{\Z}{\mathbb{Z}}
\providecommand{\Leg}[2]{\genfrac{(}{)}{}{}{#1}{#2}}


\begin{document}

\begin{center}
{\bf \Large CPSC 418 / MATH 318 --- Introduction to Cryptography

ASSIGNMENT 3 }
\end{center}

\hrule 	

\textbf{Name:} Harjee Johal \\
\textbf{Student ID:} 30000668

\medskip \hrule

\begin{enumerate} \itemsep 20pt

\stepcounter{problem}
\item[] \textbf{Problem \theproblem} --- Flawed MAC designs, 13 marks

\begin{enumerate}

\item The message $M_2$ is of the form $M_2 = M_1 || X$, where $X$ is an arbitrary $n$-bit block. From
the question description we also know that $M_1 = P_1 || P_2 || \ldots || P_L$, where $P_1, P_2, \ldots, P_L$ are
each $n$-bit blocks. In this case, 

\begin{align*}
    \PHMAC_K{(M_2)} &= ITHASH(K || M_1 || X) \\
    &= ITHASH(K || P_1 || P_2 || \ldots || P_L || X)
\end{align*}

From the algorithmic description of the ITHASH function, we can see that the result of a given iteration ITHASH$(K || P_1 || P_2 || \ldots || P_{i - 1} || P_i)$, is equal to $f(H, P_i)$ where $f$ is a public compression function, and $H$ is the value of ITHASH computed in the previous iteration: ITHASH$(K || P_1 || P_2 || \ldots || P_{i - 1})$, and $P_i$ is the $i^{th}$ $n$-bit block of the message being hashed. \\

From this, we can see that:

\begin{align*}
    \PHMAC_K{(M_2)} &= ITHASH(K || M2) \\
    &= ITHASH(K || P_1 || P_2 || \ldots || P_L || X) \\
    &= f(ITHASH(K || P_1 || P_2 || \ldots || P_L), X) \\
    &= f(ITHASH(K || M_1), X) \\
    &= f(\PHMAC_K{(M_1)}, X)
\end{align*}

Since both $\PHMAC_K{(M_1)}$ and $X$ are known, and $f$ is a publicly accessible function, that means that we can compute $\PHMAC_K{(M_2)}$ as $f(\PHMAC_K{(M_1)}, X)$, which does not require any knowledge of the value of $K$. Thus, the computational resistance of PHMAC is thwarted. \\


\item % part (b)

Since ITHASH is defined as being not weakly collision resistant, that means that given a pair ($M$, ITHASH($M$)), it's possible to find another pair ($M'$, ITHASH($M'$)) such that 

\begin{align*}
    ITHASH(M) = ITHASH(M')
\end{align*} and 

\begin{align*}
    M \neq M'
\end{align*}

Since the compression function used in ITHASH is public, that means that it's possible to compute ITHASH($M_1$). Then, given ($M_1$, ITHASH($M_1$)), we can find a value $M_2$ such that $M_1 \neq M_2$ and ITHASH($M_1$) = ITHASH($M_2$). By definition,

\begin{align*}
    \AHMAC_K{(M_2)} = ITHASH(M_2 || K)
\end{align*}

In the previous part, we demonstrated that 

\begin{align*}
    ITHASH(K || P_1 || P_2 || \ldots || P_{i - 1} || P_i) = f(ITHASH(K || P_1 || P_2 || \ldots || P_{i - 1}), P_i)
\end{align*}

We know that the key $K$ is also $n$-bits long. That means that $\AHMAC_K{(M_2)}$ can be re-written as:

\begin{align*}
    \AHMAC_K{(M_2)} &= ITHASH(M_2 || K) \\
    &= f(ITHASH(M_2), K)
\end{align*}

By definition, ITHASH($M_2$) = ITHASH($M_1$). This means that $\AHMAC_K{(M_2)}$ can once again be re-written as:

\begin{align*}
    \AHMAC_K{(M_2)} &= f(ITHASH(M_1), K) \\
    &= ITHASH(M_1 || K)
\end{align*}

By definition, $\AHMAC_K{(M_1)} = ITHASH(M_1 || K)$. This means that $\AHMAC_K{(M_1)} = \AHMAC_K{(M_2)}$. Thus, we have demonstrated that given a message/ AHMAC pair ($M_1$, AHMAC($M_1$)), it's possible to find a message/ AHMAC pair ($M_2$, AHMAC($M_2$)) without knowledge of $K$. Thus, computational resistance has been defeated in this case.

\end{enumerate}

\newpage

\stepcounter{problem}
\item[] \textbf{Problem \theproblem} --- Fast RSA decryption using Chinese remaindering, 8
    marks)

We are told that $d_p \equiv d \quad \mod{p - 1}$. This means that $d_p$ can be written in the form: $d_p = d + j(p-1)$, where $j$ is an integer. We are also told that $d_q \equiv d \quad \mod{q - 1}$. This means that $d_q$ can be written in the form: $d_q = d + k(q-1)$, where $k$ is an integer. \\ 

Next, we are told that $M_p \equiv C^{d_p} \quad \mod{p}$. Since we know that $d_p = d + j(p-1)$, that means that we can re-write this as: $M_p \equiv C^{d + j(p-1)} \quad \mod{p}$. We can manipulate this using power rules to obtain:

\begin{align*}
    &C^{d + j(p-1)} \quad \mod{p} \\
    \equiv &C^d C^{j(p-1)} \quad \mod{p} \\
    \equiv &C^d (C^{p-1})^j \quad \mod{p}
\end{align*}

Since $p$ is a prime, that means that $\phi{(p)} = p - 1$. Furthermore, since we're told that $gcd(n, C) = 1$, that means that $C$ and $n$ share no prime factors. Since $n$'s prime factors are $p$ and $q$, that means that $gcd(p, C)$ and $gcd(q, C)$ must also be 1. Since $\phi{(p)} = p - 1$ and $gcd(p, C) = 1$, we can apply Euler's theorem on $C^{p - 1}$, meaning that $C^{p - 1} \equiv 1 \quad \mod{p}$. Using this, we can further simplify $M_p$:

\begin{align*}
    &C^d (C^{p-1})^j \quad \mod{p} \\
    \equiv &C^d (1)^j \quad \mod{p} \\
    \equiv &C^d (1) \quad \mod{p} \\
    \equiv &C^d \quad \mod{p}
\end{align*}

Thus, we can see that $M_p \equiv C^d \quad \mod{p}$. That means that $M_p = C^d + sp$, where $s$ is an integer. \\

Furthermore, we are told that $M_q \equiv C^{d_q} \quad \mod{q}$. Since we know that $d_q = d + k(q-1)$, that means that we can re-write this as: $M_q \equiv C^{d + k(q-1)} \quad \mod{q}$. We can manipulate this using power rules to obtain:

\begin{align*}
    &C^{d + k(q-1)} \quad \mod{q} \\
    \equiv &C^d C^{k(q-1)} \quad \mod{q} \\
    \equiv &C^d (C^{q-1})^k \quad \mod{q}
\end{align*}

Since $q$ is a prime, that means that $\phi{(q)} = q - 1$. Earlier, we demonstrated that since $gcd(n, C) = 1$, then $gcd(q, C)$ must also equal 1. Since $\phi{(q)} = q - 1$ and $gcd(q, C) = 1$, we can apply Euler's theorem on $C^{q - 1}$, meaning that $C^{q - 1} \equiv 1 \quad \mod{q}$. Using this, we can further simplify $M_q$:

\begin{align*}
    &C^d (C^{q-1})^k \quad \mod{q} \\
    \equiv &C^d (1)^k \quad \mod{q} \\
    \equiv &C^d (1) \quad \mod{q} \\
    \equiv &C^d \quad \mod{q}
\end{align*}


Thus, we can see that $M_q \equiv C^d \quad \mod{q}$. That means that $M_q = C^d + tq$, where $t$ is an integer. \\

Lastly, we are told that $M' \equiv pxM_q + qyM_p \quad \mod{n}$, where $M'$ is the value derived from this version of RSA decryption. We can substitute $M_p$ with $C^d + sp$ and $M_q$ with $C^d + tq$ to obtain:

\begin{align*}
    M' &\equiv pxM_q + qyM_p \quad \mod{n} \\
    &\equiv px(C^d + tq) + qy(C^d + sp) \quad \mod{n} \\
    &\equiv pxC^d + pqtx + qyC^d + pqsy \quad \mod{n} \\
    &\equiv C^d(px + qy) + pq(tx + sy) \quad \mod{n}
\end{align*}

Where $tx + sy$ is an integer. We know that $n = pq$, and from step 3 of the algorithm we know that $px + qy = 1$. Using this information, we obtain:

\begin{align*}
    M' &\equiv C^d(px + qy) + pq(tx + sy) \quad \mod{n} \\
    &\equiv C^d + n(tx + sy) \quad \mod{n} \\
    &\equiv C^d \quad \mod{n}
\end{align*}

We can remove $n(tx + sy)$ from the expression since it's a multiple of $n$, and the modular arithmetic is being done modulus $n$. Therefore, we obtain $M' \equiv C^d \quad \mod{n} \equiv M^{ed} \quad \mod{n}$. By definition in RSA, $M^{ed} \equiv M \mod{n}$. Therefore, that means that $M' \equiv M^{ed} \quad \mod{n} \equiv M \quad \mod{n}$. Therefore, the $M'$ that we obtain using this method of decryption is the same as the $M$ determined during "normal" RSA decryption.

\newpage

\stepcounter{problem}
\item[] \textbf{Problem \theproblem} ---  RSA primes too close together, 18 marks)


\begin{enumerate}
\item % part (a)
We are told that $y > 0$. We are also told that $x + y = p$. This means that $y = p - x$. Since $y > 0$, then that means that $p - x > 0$ as well. Since $p - x > 0$, then $p > x$. \\

We are told that $n = x^2 - y^2$. This means that $y^2 = x^2 - n$. Since $y > 0$, that means that $y^2 > 0$. Since $y^2 = x^2 - n$, and $y^2 > 0$, then that means $x^2 - n > 0$ as well. If $x^2 - n > 0$, then $x^2 > n$, which means that $x > \sqrt{n}$. \\

From this, we can see that $p > x$ and that $x > \sqrt{n}$. Thus, we can see that $p > x > \sqrt{n}$.


\item  % part (b)
We're told that in the Fermat factorization algorithm there is a \textit{while} loop that computes both $a = a + 1$ and $b = \sqrt{a^2 - n}$. This loop continues while the computed value of $b$ isn't an integer. \\

We know that $n = x^2 - y^2$. Therefore, we can re-write this equation as: $b = \sqrt{a^2 - (x^2 - y^2)}$. When $a = x$, then we get:

\begin{align*}
    b &= \sqrt{a^2 + y^2 - x^2} \\
    b &= \sqrt{(x)^2 + y^2 - x^2} \\
    b &= \sqrt{y^2} \\
    b &= y
\end{align*}

Thus, when $a = x$, then $b = y$. Since by definition $y$ is an integer, that means that the \textit{while} loop will terminate when $a = x$. \\

We're also told that the algorithm outputs $a - b$ once it's completed. Since the \textit{while} loop terminates when $a = x$, and $b = y$ when $a = x$, then that means $a - b = x - y$. By definition, $q = x - y$, meaning that when this algorithm terminates, it outputs $q$. \\

We can also show that the \textit{while} loop won't terminate for a value of $a$ less than $x$ using a proof by contradiction. Suppose that the loop does terminate for some integer $a < x$. Then that means that for that value of $a$, we get $b = \sqrt{a^2 - n}$, where $b$ is an integer. We can re-arrange the equation $b = \sqrt{a^2 - n}$ to obtain:

\begin{align*}
    b &= \sqrt{a^2 - n} \\
    b^2 &= a^2 - n \\
    n &= a^2 - b^2 \\
    n &= (a + b)(a - b)
\end{align*}

From this, we can see that $n = (a + b)(a - b)$. Since $n = pq$ and $n > p > q > 0$, then that means $p = a + b$ and $q = a - b$. From these two equations, we can see that $a = \frac{p + q}{2}$. By definition, $x = \frac{p + q}{2}$. Therefore, that would mean that $a = x$. However, we defined $a$ such that $a < x$. This is a contradiction. Therefore, this shows that the loop does not terminate for any value of $a < x$.

\item  % part (c)
The value of $a$ is initialized as $a = \lceil \sqrt{n} \rceil$. During the first iteration of the loop, the algorithm first performs $a = a + 1$, and then computes $b = \sqrt{a^2 - n}$. It continues to do this until it finally reaches $a = x$. This means that the first iteration of the loop is performed with $a = \lceil \sqrt{n} \rceil + 1$, and the last iteration of the loop is done with $a = x$. The number of iterations between the first and last values of $a$ is thus $x - \lceil \sqrt{n} \rceil$. The last iteration of the loop is after $a = x$. The condition at the top of the \textit{while} loop is checked one last time. Since $b$ is an integer when $a = x$, that means that the condition at the top of the \textit{while} loop will not be satisfied on the last iteration, meaning that the \textit{while} loop will be skipped. However, this iteration is still counted. Therefore, there are $x - \lceil \sqrt{n} \rceil$ iterations where the loop is entered, and $1$ iteration where the loop is not entered. Therefore, there are $x - \lceil \sqrt{n} \rceil + 1$ iterations overall.
\\

\item  % part (d)
We are asked to prove that $x - \lceil \sqrt{n} \rceil < \frac{y^2}{2\sqrt{n}}$. We know that $n = x^2 - y^2$. We can rearrange this to get $y^2 = x^2 - n = (x + \sqrt{n})(x - \sqrt{n})$. We can re-arrange this equation to get: $(x - \sqrt{n}) = \frac{y^2}{x + \sqrt{n}}$. \\

In part (a), we showed that $x > \sqrt{n}$. From this, we can see that $x + \sqrt{n} > 2\sqrt{n}$. Since $2\sqrt{n} < x + \sqrt{n}$, that means that $\frac{y^2}{2\sqrt{n}} > \frac{y^2}{x + \sqrt{n}}$. Since $(x - \sqrt{n}) = \frac{y^2}{x + \sqrt{n}}$, that means that $(x - \sqrt{n}) < \frac{y^2}{2\sqrt{n}}$. \\

By definition, $\lceil \sqrt{n} \rceil \geq \sqrt{n}$. Since $\lceil \sqrt{n} \rceil \geq \sqrt{n}$, then that means $x - \sqrt{n} \geq x - \lceil \sqrt{n} \rceil$. We can use this to show that since $(x - \sqrt{n}) < \frac{y^2}{2\sqrt{n}}$, that means $(x - \lceil \sqrt{n} \rceil) < \frac{y^2}{2\sqrt{n}}$. Thus, we have proven that $x - \lceil \sqrt{n} \rceil < \frac{y^2}{2\sqrt{n}}$.
\\

\item  % part (e)
Suppose that $p - q < 2B\sqrt[4]{n}$. We can re-arrange this to obtain:

\begin{align*}
    \frac{p - q}{2} &< B\sqrt[4]{n} \\
    y &< B\sqrt[4]{n} \quad \quad (y = \frac{p - q}{2}) \\
    y^2 &< B^2\sqrt{n} \\
    \frac{y^2}{\sqrt{n}} &< B^2 \\
    \frac{y^2}{2\sqrt{n}} &< \frac{B^2}{2}
\end{align*}

Thus, we can see that $\frac{y^2}{2\sqrt{n}} < \frac{B^2}{2}$. In part (d), we showed that $x - \lceil \sqrt{n} \rceil < \frac{y^2}{2\sqrt{n}}$. Using this statement, we can see that $x - \lceil \sqrt{n} \rceil < \frac{B^2}{2}$. Since $x - \lceil \sqrt{n} \rceil < \frac{B^2}{2}$, then that means that $x - \lceil \sqrt{n} \rceil + 1 < \frac{B^2}{2} + 1$. We showed in part (c) that the number of loops iterations performed by Fermat's Factorization algorithm to factor $n$ is also $x - \lceil \sqrt{n} \rceil + 1$. Therefore, we can see that algorithm factors $n$ after at most $\frac{B^2}{2} + 1$ iterations. 

\end{enumerate}

\newpage

\stepcounter{problem}
\item[] \textbf{Problem \theproblem} -- The El Gamal public key cryptosystem is not
    semantically secure, 12 marks

We are asked to prove that El Gamal is not semantically secure. We do this by proving 6 assertions. The assertions are as follows: \\

\textbf{\textit{Assertion 1:}} If $\Leg{y}{p} = 1$ and $\Leg{C_2}{p} = 1$, then $C = E(M_1)$:

Since $C_2 \equiv My^k \mod{p}$, this implies that $\Leg{C_2}{p} = \Leg{My^k}{p} = \Leg{M}{p} \Leg{y}{p}^k$. Since $\Leg{y}{p} = 1$ and $\Leg{C_2}{p} = 1$, we can substitute them into the equation to get: $1 = \Leg{M}{p} (1)^k = \Leg{M}{p}$. Since $\Leg{M}{p} = 1$, then that means that $M$ is a quadratic residue modulo $p$. We are told in the question that $M_1$ is a quadratic residue modulo $p$, and $M_2$ is not a quadratic residue modulo $p$. Therefore since the $M$ in $C_2 \equiv My^k \mod{p}$ is a quadratic residue modulo $p$, it must be $M_1$, meaning that in this case $C = E(M_1)$. Thus the assertion is true. \\ \\


\textbf{\textit{Assertion 2:}} If $\Leg{y}{p} = 1$ and $\Leg{C_2}{p} = -1$, then $C = E(M_2)$:

Since $C_2 \equiv My^k \mod{p}$, this implies that $\Leg{C_2}{p} = \Leg{My^k}{p} = \Leg{M}{p} \Leg{y}{p}^k$. Since $\Leg{y}{p} = 1$ and $\Leg{C_2}{p} = -1$, we can substitute them into the equation to get: $-1 = \Leg{M}{p} (1)^k = \Leg{M}{p}$. Since $\Leg{M}{p} = -1$, then that means that $M$ is not a quadratic residue modulo $p$. We are told in the question that $M_1$ is a quadratic residue modulo $p$, and $M_2$ is not a quadratic residue modulo $p$. Therefore since the $M$ in $C_2 \equiv My^k \mod{p}$ is not a quadratic residue modulo $p$, it must be $M_2$, meaning that in this case $C = E(M_2)$. Thus the assertion is true. \\ \\


\textbf{\textit{Assertion 3:}} If $\Leg{y}{p} = -1$ and $\Leg{C_1}{p} = 1$ and $\Leg{C_2}{p} = 1$, then $C = E(M_1)$:

Since $y \equiv g^x \mod{p}$, then that means that $\Leg{y}{p} = \Leg{g}{p}^x$. Since we know that $\Leg{y}{p} = -1$, that means that $-1 = \Leg{g}{p}^x$. From this, we can see that $\Leg{g}{p} = -1$, and $x$ must be an odd integer. Next, we see that $C_1 \equiv g^k \mod{p}$, meaning that $\Leg{C_1}{p} = \Leg{g}{p}^k$. We know that $\Leg{C_1}{p} = 1$ and $\Leg{g}{p} = -1$, meaning that $1 = (-1)^k$. Therefore, $k$ must be an even number. \\

Next, we can see that since $C_2 \equiv My^k \mod{p}$, then $\Leg{C_2}{p} = \Leg{M}{p} \Leg{y}{p}^k$. We know that $\Leg{C_2}{p} = 1$ and that $\Leg{y}{p} = -1$ and that $k$ is an even number. Therefore, we can see that $1 = \Leg{M}{p} (-1)^{even} = \Leg{M}{p}$. Since $\Leg{M}{p} = 1$, then that means that $M$ is a quadratic residue modulo $p$. We know that $M_1$ is a quadratic residue modulo $p$, and $M_2$ is not a quadratic residue modulo $p$. Therefore since the $M$ in $C_2 \equiv My^k \mod{p}$ is a quadratic residue modulo $p$, it must be $M_1$, meaning that in this case $C = E(M_1)$. Thus the assertion is true. \\ \\


\textbf{\textit{Assertion 4:}} If $\Leg{y}{p} = -1$ and $\Leg{C_1}{p} = 1$ and $\Leg{C_2}{p} = -1$, then $C = E(M_2)$:

Since $y \equiv g^x \mod{p}$, then that means that $\Leg{y}{p} = \Leg{g}{p}^x$. Since we know that $\Leg{y}{p} = -1$, that means that $-1 = \Leg{g}{p}^x$. From this, we can see that $\Leg{g}{p} = -1$, and $x$ must be an odd integer. Next, we see that $C_1 \equiv g^k \mod{p}$, meaning that $\Leg{C_1}{p} = \Leg{g}{p}^k$. We know that $\Leg{C_1}{p} = 1$ and $\Leg{g}{p} = -1$, meaning that $1 = (-1)^k$. Therefore, $k$ must be an even number. \\

Next, we can see that since $C_2 \equiv My^k \mod{p}$, then $\Leg{C_2}{p} = \Leg{M}{p} \Leg{y}{p}^k$. We know that $\Leg{C_2}{p} = -1$ and that $\Leg{y}{p} = -1$ and that $k$ is an even number. Therefore, we can see that $-1 = \Leg{M}{p} (-1)^{even} = \Leg{M}{p}$. Since $\Leg{M}{p} = -1$, then that means that $M$ is not a quadratic residue modulo $p$. We know that $M_1$ is a quadratic residue modulo $p$, and $M_2$ is not a quadratic residue modulo $p$. Therefore since the $M$ in $C_2 \equiv My^k \mod{p}$ is not a quadratic residue modulo $p$, it must be $M_2$, meaning that in this case $C = E(M_2)$. Thus the assertion is true. \\ \\


\textbf{\textit{Assertion 5:}} If $\Leg{y}{p} = -1$ and $\Leg{C_1}{p} = -1$ and $\Leg{C_2}{p} = 1$, then $C = E(M_2)$:

Since $y \equiv g^x \mod{p}$, then that means that $\Leg{y}{p} = \Leg{g}{p}^x$. Since we know that $\Leg{y}{p} = -1$, that means that $-1 = \Leg{g}{p}^x$. From this, we can see that $\Leg{g}{p} = -1$, and $x$ must be an odd integer. Next, we see that $C_1 \equiv g^k \mod{p}$, meaning that $\Leg{C_1}{p} = \Leg{g}{p}^k$. We know that $\Leg{C_1}{p} = -1$ and $\Leg{g}{p} = -1$, meaning that $-1 = (-1)^k$. Therefore, $k$ must be an odd number. \\

Next, we can see that since $C_2 \equiv My^k \mod{p}$, then $\Leg{C_2}{p} = \Leg{M}{p} \Leg{y}{p}^k$. We know that $\Leg{C_2}{p} = 1$ and that $\Leg{y}{p} = -1$ and that $k$ is an odd number. Therefore, we can see that $1 = \Leg{M}{p} (-1)^{odd} = -\Leg{M}{p}$, meaning that $\Leg{M}{p} = -1$. Since $\Leg{M}{p} = -1$, then that means that $M$ is not a quadratic residue modulo $p$. We know that $M_1$ is a quadratic residue modulo $p$, and $M_2$ is not a quadratic residue modulo $p$. Therefore since the $M$ in $C_2 \equiv My^k \mod{p}$ is not a quadratic residue modulo $p$, it must be $M_2$, meaning that in this case $C = E(M_2)$. Thus the assertion is true. \\ \\


\textbf{\textit{Assertion 6:}} If $\Leg{y}{p} = -1$ and $\Leg{C_1}{p} = -1$ and $\Leg{C_2}{p} = -1$, then $C = E(M_1)$:

Since $y \equiv g^x \mod{p}$, then that means that $\Leg{y}{p} = \Leg{g}{p}^x$. Since we know that $\Leg{y}{p} = -1$, that means that $-1 = \Leg{g}{p}^x$. From this, we can see that $\Leg{g}{p} = -1$, and $x$ must be an odd integer. Next, we see that $C_1 \equiv g^k \mod{p}$, meaning that $\Leg{C_1}{p} = \Leg{g}{p}^k$. We know that $\Leg{C_1}{p} = -1$ and $\Leg{g}{p} = -1$, meaning that $-1 = (-1)^k$. Therefore, $k$ must be an odd number. \\

Next, we can see that since $C_2 \equiv My^k \mod{p}$, then $\Leg{C_2}{p} = \Leg{M}{p} \Leg{y}{p}^k$. We know that $\Leg{C_2}{p} = -1$ and that $\Leg{y}{p} = -1$ and that $k$ is an odd number. Therefore, we can see that $-1 = \Leg{M}{p} (-1)^{odd} = -\Leg{M}{p}$, meaning that $\Leg{M}{p} = 1$. Since $\Leg{M}{p} = 1$, then that means that $M$ is a quadratic residue modulo $p$. We know that $M_1$ is a quadratic residue modulo $p$, and $M_2$ is not a quadratic residue modulo $p$. Therefore since the $M$ in $C_2 \equiv My^k \mod{p}$ is a quadratic residue modulo $p$, it must be $M_1$, meaning that in this case $C = E(M_1)$. Thus the assertion is true. \\

We've proved that all of Mallory's assertions are true. Therefore, El Gamal is not semantically secure.

\newpage


\stepcounter{problem}
\item[] \textbf{Problem \theproblem} --- An IND-CPA, but not IND-CCA secure version of RSA, 12
    marks
    
    We are asked to show that the version of RSA specified in this question is not IND-CCA secure. We start by choosing two different plaintexts, $M_1$ and $M_2$, and receive a ciphertext $C$ that is an encryption of one of them. The ciphertext can be represented as:
    
    \begin{align*}
        C = (s||t) = (r^e (mod \quad n) || H(r) \oplus M_i)
    \end{align*}
    
    where i = 1 or i = 2. The value $r$ is a random $k$-bit value such that $r < n$. In this case, $n$ also has $k$-bits. $H$ is a public random function that maps $\{0, 1\}^k$ to $\{0, 1\}^m$, where $m$ is the bit-length of the message being encrypted. \\
    
    We then compute a new ciphertext, $C'$ from $C$, such that:
    
    \begin{align*}
        C' = (s||t')
    \end{align*}
    
    where $t' = t \oplus M_1$. From here, we have two cases: \\
    
    \textbf{\textit{Case 1:}} $C$ is an encryption of $M_1$.
    
    In the question, we are told that decryption is done via $M \equiv H(s^d (mod \quad n)) \oplus t$. The decryption of $C'$ would therefore be: $M \equiv H(s^d (mod \quad n)) \oplus t'$. Since $t' = t \oplus M_1$, and in this case we know that $t = H(r) \oplus M_1$, that means that $t' = H(r) \oplus M_1 \oplus M_1$. A number XOR'd with itself equals zero, meaning that $t' = H(r) \oplus 0 = H(r)$. Therefore, the decryption of $C'$ can be simplified to: 
    
    \begin{align*}
        M &\equiv H(s^d (mod \quad n)) \oplus t' \\
        &\equiv H(s^d (mod \quad n)) \oplus H(r)
    \end{align*}
    
    We know that $s = r^e$, meaning $s^d = r^{ed}$. By the definition of $e$ and $d$, we know that $ed = 1 + k\phi{(n)}$, meaning that $r^{ed} = r^{1 + k\phi{(n)}} = r (r^{\phi{(n)}})$. We can use Euler's theorem to show that $r^{\phi{(n)}} \equiv 1 \mod{n}$ (the probability of $gcd(r, n) \neq 1$ is very low). Thus, we can see that $r^{ed} \equiv r \mod{n}$. Therefore, $H(s^d (mod \quad n)) = H(r)$. Thus, we can once again modify the decryption process to see that:
    
    \begin{align*}
        M &\equiv H(s^d (mod \quad n)) \oplus H(r) \\
        &= H(r) \oplus H(r) \\
        &= 0
    \end{align*}
    
    Therefore, when $C$ is an encryption of $M_1$, then the decryption of $C'$ is 0. \\ \\
    
    
    \textbf{\textit{Case 2:}} $C$ is an encryption of $M_2$.
    
    In the question, we are told that decryption is done via $M \equiv H(s^d (mod \quad n)) \oplus t$. The decryption of $C'$ would therefore be: $M \equiv H(s^d (mod \quad n)) \oplus t'$. Since $t' = t \oplus M_1$, and in this case we know that $t = H(r) \oplus M_2$, that means that $t' = H(r) \oplus M_1 \oplus M_2$. Therefore, the decryption of $C'$ can be re-written as: 
    
    \begin{align*}
        M &\equiv H(s^d (mod \quad n)) \oplus t' \\
        &\equiv H(s^d (mod \quad n)) \oplus H(r) \oplus M_1 \oplus M_2
    \end{align*}
    
    From the previous case, we showed that $H(s^d (mod \quad n)) = H(r)$. Thus, we can once again modify the decryption process to see that:
    
    \begin{align*}
        M &\equiv H(s^d (mod \quad n)) \oplus H(r) \oplus M_1 \oplus M_2 \\
        &= H(r) \oplus H(r) \oplus M_1 \oplus M_2 \\
        &= M_1 \oplus M_2
    \end{align*}
    
    Therefore, when $C$ is an encryption of $M_2$, then the decryption of $C'$ is $M_1 \oplus M_2$. Mallory can easily compute this value, since she selects both $M_1$ and $M_2$. \\
    
    From this, we can see that Mallory has a method to easily identify whether $C$ is the ciphertext of $M_1$ or $M_2$ based on the decryption of $C'$. If the decryption of $C'$ is 0, then $C$ is an encryption of $M_1$. Otherwise, if the decryption of $C'$ is $M_1 \oplus M_2$, then $C$ is an encryption of $M_2$.
    

\newpage

\stepcounter{problem}
\item[]  \textbf{Problem \theproblem} --- An attack on RSA with small decryption exponent, 25
    marks

\begin{enumerate}

\item % part (a)
By definition, $e, d > 1$. The minimum value that $ed$ could possibly take occurs when $e = d = 2$, where we get $ed = 4$. This means that it's not possible to pick values for $e, d$ such that $ed = 1$. Therefore, the minimum value that $ed$ can take on such that $ed \equiv 1 \mod{\phi(n)}$ is $ed = 1 + \phi{(n)}$. In this case, $k = 1$, since $ed = 1 + (1)\phi{(n)}$. Thus, we have demonstrated that $k \geq 1$. \\

We can show that $k < d$ through a proof by contradiction. Suppose that $k \geq d$. We know that since $ed \equiv 1 \mod{\phi(n)}$, then $ed = 1 + k\phi(n)$. This can be re-arranged to obtain $ed - k\phi(n) = 1$. Since $k \geq d$, then that means that in order for this equation to hold, $e > \phi(n)$ must also be true. This is because if $e \leq \phi(n)$ that would imply that $ed \leq k\phi(n)$, which would make $ed - k\phi(n) = 1$ impossible. However, $e > \phi(n)$ is not possible, since by definition $1 < e < \phi(n)$. This is a contradiction, meaning that it's not possible to have $k \geq d$. Thus, it must be the case that $k < d$. \\

From the previous two explanations, we can see that $1 \leq k < d$. \\

Next, we must prove that $gcd(d, k) = 1$. We can prove this by using the properties of modular inversion. By definition, a number $a$ has an inverse modulo $m$, meaning that there exists some integer $x$ such that $ax \equiv 1 \mod{m}$, if and only if $gcd(a, m) = 1$. Furthermore, if $ax \equiv 1 \mod{m}$, then that means $ax - my = 1$, for some integer $y$. \\

Now, let's revisit $ed = 1 + k\phi(n)$. We can re-write this as $ed - k\phi(n) = 1$, which is similar to the form $ax - my = 1$. From this, we can draw two potential conclusions. The first conclusion is that $ed \equiv 1 \mod{\phi(n)}$, meaning that $d$ has an inverse modulo $\phi(n)$, $e$. Thus, $gcd(d, \phi(n))$. 
\\ \\
The second conclusion is that $ed \equiv 1 \mod{k}$. This holds because we can write 

\begin{align*}
    ed - km = 1
\end{align*}

for some integer $m$ (in this case $m = \phi(n)$). Therefore, since $ed \equiv 1 \mod{k}$ that means that $e$ is $d$'s inverse modulo $k$. Since $d$ has an inverse modulo $k$, that means that $gcd(d, k) = 1$ by the definition of a modular inverse.
\\

\item % part (b)
In this question, we are asked to prove that $2 \leq n - \phi(n) < 3\sqrt{n}$. Since $n = pq$ and $\phi(n) = (p-1)(q-1)$, we can re-write $n - \phi(n)$ as:

\begin{align*}
    n - \phi(n) &= pq - (p - 1)(q - 1) \\
    &= pq - (pq - p - q + 1) \\
    &= pq - pq + p + q - 1 \\
    n - \phi(n) &= p + q - 1
\end{align*}

Now that we've established this, we can find the minimum value of $n - \phi(n)$ by finding the minimum values for $p$ and $q$. We are told that both $p$ and $q$ are odd primes, and that $q < p < 2q$. The two smallest values that satisfy these conditions are $q = 3$ and $p = 5$. We then plug these values into the equation:

\begin{align*}
    n - \phi(n) &= p + q - 1 \\
    &= 5 + 3 - 1 \\
    &= 7
\end{align*}

Therefore, the minimum value for $n - \phi(n)$ under these constraints is 7. From inspection, we can see that $7 \geq 2$, meaning that we have demonstrated that $n - \phi(n) \geq 2$.
\\

Next, we must prove that $n - \phi(n) < 3\sqrt{n}$. We've already shown that:

\begin{align*}
    n - \phi(n) = p + q - 1
\end{align*}

We also know that $p$ and $q$ must be odd primes such that $q < p < 2q$. Since we must pick $p < 2q$, that means that our value for $n - \phi(n)$ is also bounded such that:

\begin{align*}
    n - \phi(n) &= p + q - 1 \\
    &< (2q) + q - 1  = 3q - 1\\
    n - \phi(n) &< 3q - 1
\end{align*}

Thus, we see that $n - \phi(n) < 3q - 1$. We know that $n = pq$. Since $p < 2q$, then that means

\begin{align*}
    n &< pq \\
    &< (2q)q \\
    &< 2q^2
\end{align*}

Thus, we know that $n < 2q^2$. From this, we can see that $\sqrt{n} < \sqrt{2}q$, and from this we see that $3\sqrt{n} < 3\sqrt{2}q$. \\

Now, we compare these two pieces of information. We know that the upper bound of $n - \phi(n)$ is $3q - 1$, and we know that the upper bound of $3\sqrt{n}$ is $3\sqrt{2}q$. We also know that the upper bound for both of these values is defined by the constraint $p < 2q$. By inspection, we can see that $3\sqrt{2}q > 3q - 1$ for all values of $q$, meaning that $3\sqrt{n} > 3q - 1$. Since we know that $n - \phi(n) < 3q - 1$ and $3q - 1 < 3\sqrt{n}$, then that must mean that $n - \phi(n) < 3\sqrt{n}$. \\

We have proven that $2 \leq n - \phi(n)$ and that $n - \phi(n) < 3\sqrt{n}$. Therefore, we have proven that $2 \leq n - \phi(n) < 3\sqrt{n}$.
\\

\item % part (c)

In this part we are asked to prove that $0 < kn - ed < 3d\sqrt{n}$. From the question description, we know that $ed = 1 + k\phi(n)$. We can use this equation to obtain:

\begin{align*}
    kn - ed &= kn - (1 + k\phi(n)) \\
    &= kn - 1 - k\phi(n) \\
    &= k(n - \phi(n)) - 1
\end{align*}

From part (b), we know that $2 \leq n - \phi(n)$. From this, we can see that if $2 \leq n - \phi(n)$, then $2k - 1 \leq k(n - \phi(n)) - 1$. From part (a), we know that $1 \leq k$. Therefore, the minimum value $k$ can take is $k = 1$. If we plug this into the inequality, we obtain: 

\begin{align*}
    2k - 1 = 2(1) - 1 = 1 \leq k(n - \phi(n)) - 1
\end{align*}

Thus, we have shown that $1 \leq k(n - \phi(n)) - 1$. From this, we can conclude that $0 < k(n - \phi(n)) - 1$ \\

Next, we must show that $k(n - \phi(n)) - 1 < 3d\sqrt{n}$. From part (b), we know that $n - \phi(n) < 3\sqrt{n}$. From this, we can see that $k(n - \phi(n)) - 1 < 3k\sqrt{n} - 1$. In part (a), we determined that $k < d$. Therefore, it must be true that $3k\sqrt{n} - 1 < 3d\sqrt{n} - 1$. Since $3d\sqrt{n} - 1 < 3d\sqrt{n}$, that means that $3k\sqrt{n} - 1 < 3d\sqrt{n}$ must also be true. Since we know $k(n - \phi(n)) - 1 < 3k\sqrt{n} - 1$, and $3k\sqrt{n} - 1 < 3d\sqrt{n}$, we can see that $k(n - \phi(n)) - 1 < 3d\sqrt{n}$. \\

Thus we have also proven both $0 < k(n - \phi(n)) - 1$ and $k(n - \phi(n)) - 1 < 3d\sqrt{n}$, meaning that we have proven:

\begin{align*}
    0 < k(n - \phi(n)) - 1 < 3d\sqrt{n}
\end{align*}

Since $k(n - \phi(n)) - 1 = kn - ed$, that means we've proven:

\begin{align*}
    0 < kn - ed < 3d\sqrt{n}
\end{align*}
\\

\item % part (d)

We are asked to show that $0 < \frac{k}{d} - \frac{e}{n} < \frac{1}{2d^2}$. From part (c), we know that

\begin{align*}
    0 < kn - ed < 3d\sqrt{n}
\end{align*}

If we divide the entire inequality by $dn$, which we can do since $0 < d, n$ by definition, we get:

\begin{align*}
    0 < \frac{k}{d} - \frac{e}{n} < \frac{3}{\sqrt{n}}
\end{align*}

By inspection, we can already see that $0 < \frac{k}{d} - \frac{e}{n}$. Thus, we must prove that $\frac{k}{d} - \frac{e}{n} < \frac{1}{2d^2}$. In the preliminary information for the question, we are told that $d < \frac{\sqrt[4]{n}}{\sqrt{6}}$. We can re-arrange this inequality as follows:

\begin{align*}
    d &< \frac{\sqrt[4]{n}}{\sqrt{6}} \\
    d^2 &< \frac{\sqrt{n}}{6} \\
    6d^2 &< \sqrt{n}
\end{align*}

From this, we can conclude that $6d^2 < \sqrt{n}$. If $6d^2 < \sqrt{n}$, then that means $\frac{3}{\sqrt{n}} < \frac{3}{6d^2}$. Since $\frac{k}{d} - \frac{e}{n} < \frac{3}{\sqrt{n}}$ and $\frac{3}{\sqrt{n}} < \frac{3}{6d^2}$, then that means $\frac{k}{d} - \frac{e}{n} < \frac{3}{6d^2}$. We can simply $\frac{3}{6d^2}$ into $\frac{1}{2d^2}$, meaning that $\frac{k}{d} - \frac{e}{n} < \frac{1}{2d^2}$.
\\
Thus, we have demonstrated that $0 < \frac{k}{d} - \frac{e}{n} < \frac{1}{2d^2}$.
\\

\item % part (e)

In this question, we are asked to prove that when we apply the Euclidean theorem to $e$ and $n$, $k = A_i$ and $d = B_i$ for some $i \in \{1, 2, \ldots, m\}$, where $A_i, B_i$ are the associated sequences defined in the question. First, let us define our positive rational number $r = \frac{a}{b} \in \mathbb{Q}$ with $a, b \in \mathbb{N}$ as $r = \frac{e}{n}$ as specified in the question. We can then use the theorem defined in the question. The theorem states if we let $r = \frac{a}{b} \in \mathbb{Q}$ and let $\frac{A}{B} \in \mathbb{Q}$ be a fraction in the lowest terms such that

\begin{align*}
    \mid r - \frac{A}{B} \mid < \frac{1}{2B^2}
\end{align*}

Then $A = A_i$ and $B = B_i$ for some $i \in \{1, 2, \ldots, m\}$. In our case, $r = \frac{e}{n}$, meaning that we must find values $A$ and $B$ such that

\begin{align*}
    \mid \frac{e}{n} - \frac{A}{B} \mid < \frac{1}{2B^2}
\end{align*}

Suppose we set $A = k$ and $B = d$. Then $\frac{A}{B} = \frac{k}{d}$. We already know that $\frac{k}{d}$ is in lowest terms, since we proved that $gcd(d, k) = 1$ in part (a). Then if we could show that

\begin{align*}
    \mid \frac{e}{n} - \frac{k}{d} \mid < \frac{1}{2d^2}
\end{align*}


we could prove that $k = A_i$ and $d = B_i$ for some $i \in \{1, 2, \ldots, m\}$. By the definition of absolute values, we know that

\begin{align*}
    \mid \frac{e}{n} - \frac{k}{d} \mid \quad = \quad \mid \frac{k}{d} - \frac{e}{n} \mid
\end{align*}

Furthermore, we know from part (d) that $\frac{k}{d} - \frac{e}{n} > 0$. From this, we can conclude that 

\begin{align*}
    \mid \frac{k}{d} - \frac{e}{n} \mid \quad = \quad \frac{k}{d} - \frac{e}{n}
\end{align*}

Which ultimately means that

\begin{align*}
   \mid \frac{e}{n} - \frac{k}{d} \mid \quad = \quad \frac{k}{d} - \frac{e}{n}
\end{align*}

Thus, the inequality in the theorem now becomes:

\begin{align*}
    \frac{k}{d} - \frac{e}{n} < \frac{1}{2d^2}
\end{align*}

We already proved that this inequality is true in part (d). Therefore, we have shown that when we let $\frac{A}{B} = \frac{k}{d}$, we get a fraction in lowest terms such that 

\begin{align*}
    \mid \frac{e}{n} - \frac{k}{d} \mid < \frac{1}{2d^2}
\end{align*}

Thus according to the theorem, that means that $k = A_i$ and $d = B_i$ for some $i \in \{1, 2, \ldots, m\}$. Therefore, we have proven what was asked of us.
\\

\item % part (f)
Lastly, we are asked to derive a procedure to find $d$, and subsequently factor $n$. In order to find $d$ we can take advantage of two things. First, we can use $ed = 1 + k\phi(n)$. If we manipulate this equation, we get:

\begin{align*}
    \phi(n) = \frac{ed - 1}{k}
\end{align*}

The next thing that we can use is the statement that we proved to be true in part (e). Namely, we can use the fact that $k = A_i$ and $d = B_i$ for some $i \in \{1, 2, \ldots, m\}$ when we apply the Euclidean algorithm to $e$ and $n$. From these two pieces of information, we can derive a procedure to find $d$. \\

First, perform the Euclidean algorithm for $e$ and $n$. Then, for each $A_i$, $B_i$ for $i \in \{1, 2, \ldots, m\}$, check to see if 

\begin{align*}
    \mid \frac{e}{n} - \frac{A_i}{B_i} \mid < \frac{1}{2B_i^2}
\end{align*}

is satisfied. If it is, then this value of $A_i$ is a candidate value for $k$, and similarly this value of $B_i$ is a candidate value for $d$. In order to confirm where these candidate values are the true values for $d$ and $k$, we feed them into the following equation:

\begin{align*}
    candidate \phi(n) = \frac{eB_i - 1}{A_i}
\end{align*}

If $A_i = k$ and $B_i = d$, then this equation merely becomes

\begin{align*}
    \phi(n) = \frac{ed - 1}{k}
\end{align*}

Therefore, if a given pair $A_i$, $B_i$ output an integer when they're passed into $candidate \phi(n) = \frac{eB_i - 1}{A_i}$, then we learn 3 things: that $A_i$ is a candidate value for $k$, $B_i$ is candidate value for $d$, and we find a candidate value for $\phi(n)$. Now that we know $n$ and we have a candidate value for $\phi(n)$, we can factor $n$ to find our potential $p$ and potential $q$ by solving the following equation that was provided to us in class:

\begin{align*}
    x^2 - (n - (candidate \phi(n)) + 1)x + n = 0
\end{align*}

for x. Assuming that $candidate\phi(n)$ actually does equal $\phi(n)$, the solutions to this equation are $p$ and $q$. This equation is designed to factor $n$, meaning that the solutions of this equation can be multiplied together such that their product is $n$. Since $n$'s prime factorization is $n = pq$, that means that if the solutions of the equation are both integers, the solutions must be $p$ and $q$. This is because by definition, the product of two primes can have no integer factors outside of those two primes, barring itself and 1. Therefore, if the two solutions are integers with a product $n$, such that the solutions aren't $1$ and $n$, they must be $p$ and $q$. This of course only happens if $candidate\phi(n) = \phi(n)$. If this is the case, then we have successfully found the value of $d$ and $\phi(n)$, and also factored $n$. If this is not the case, then we must go back to the beginning of the algorithm and check the next ($A_i$, $B_i$) pair to see if

\begin{align*}
    \mid \frac{e}{n} - \frac{A_i}{B_i} \mid < \frac{1}{2B_i^2}
\end{align*}

again and repeat the process. We repeat this process until we finally find an ($A_i$, $B_i$) pair that gives us the true value of $\phi(n)$ and factors $n$ to find $p$ and $q$. We know that this will eventually happen, since we proved in part (e) that $k = A_i$ and $d = B_i$ for some $i \in \{1, 2, \ldots, m\}$. \\

We know that the number of computations is small, since there are only $m$ pairs of numbers that need to be to be checked: the pairs ($A_i$, $B_i$) for $i \in \{1, 2, \ldots, m\}$ that are generated when the Euclidean algorithm is applied to $e$ and $n$. These can all easily be found. Thus, in the worst case case, we would have to compute, 

\begin{align*}
    \mid \frac{e}{n} - \frac{A_i}{B_i} \mid < \frac{1}{2B_i^2}
\end{align*}

$m$ times, compute 

\begin{align*}
    candidate \phi(n) = \frac{eB_i - 1}{A_i}
\end{align*}

$m$ times, and solve 

\begin{align*}
    x^2 - (n - (candidate \phi(n)) + 1)x + n = 0
\end{align*}

$m$ times. Since the number of computations is bounded by $m$, we can see that the number of computations will be relatively small.

\end{enumerate}

\newpage

\stepcounter{problem}
\item[] \textbf{Problem \theproblem} --- Universal forgery attack on the El Gamal signature
    scheme, 12 marks)

\begin{enumerate}

\item % part (a)
In this question, we are asked to prove that $R \equiv ru \quad (\mod{p - 1})$, and to also show that $y^R \equiv y^{ru} \quad (\mod{p})$. By definition, we know that $R \equiv rup - r(p - 1) (\mod{p(p-1)})$. From this, we can see that 

\begin{align*}
    R &= rup - r(p - 1) + ap(p-1) \\
    &= rup + (ap - r)(p-1)
\end{align*}

For some integer $a$, where $ap - r$ is an integer. Thus, we when attempt to find $R \quad (\mod{p-1})$, we find that

\begin{align*}
    R &\equiv rup + (ap - r)(p - 1) \quad (\mod{p - 1}) \\
    &\equiv rup \quad (\mod{p - 1})
\end{align*}

Since we're performing arithmetic modulo $p - 1$, we can remove $(ap - r)(p - 1)$, since it's an integer multiple of $p - 1$. Next, we can see that $p = (p - 1) + 1$. If we re-write it as such, we get

\begin{align*}
    R &\equiv rup \quad (\mod{p - 1}) \\
    &\equiv ru((p - 1) + 1) \quad (\mod{p - 1}) \\
    &\equiv ru(p - 1) + ru \quad (\mod{p - 1}) \\
    &\equiv ru \quad (\mod{p - 1})
\end{align*}

Once again, we can remove $ru(p - 1)$ since it's an integer multiple of $p - 1$. Thus, we have proved that $R \equiv ru \quad (\mod{p - 1})$. \\

Next, we must prove that $y^R \equiv y^{ru} \quad (\mod{p})$. We know that $R \equiv ru \quad (\mod{p - 1})$, that means that

\begin{align*}
    R = ru + b(p - 1)
\end{align*}

for some integer $b$. Knowing this, we can re-write $y^R \quad (\mod{p})$:

\begin{align*}
    y^R &\equiv y^{ru + b(p - 1)} \quad (\mod{p}) \\
    &\equiv (y^{ru}) (y^{p - 1})^b \quad (\mod{p})
\end{align*}

From the question description, we know that $y \equiv g^x \quad (\mod{p})$. Therefore, we can further modify $y^R \quad (\mod{p})$ to get

\begin{align*}
    y^R &\equiv (y^{ru}) ((g^x)^{p - 1})^b \quad (\mod{p}) \\
    &\equiv (y^{ru}) (g^{p - 1})^{bx} \quad (\mod{p})
\end{align*}

Since $p$ is a prime, and $g$ is a primitive root of $p$, then by Fermat's Little Theorem, $g^{p - 1} \equiv 1 \quad (\mod{p})$. With this final piece of information, we can manipulate $y^R$ to obtain

\begin{align*}
    y^R &\equiv (y^{ru}) (g^{p - 1})^{bx} \quad (\mod{p}) \\
    &\equiv (y^{ru}) (1)^{bx} \quad (\mod{p}) \\
    &\equiv y^{ru} \quad (\mod{p})
\end{align*}

Therefore, we have demonstrated that $y^R \equiv y^{ru} \quad (\mod{p})$.
\\

\item % part (b)

Next, we are asked to prove that $R^s \equiv r^{su} \quad (\mod{p})$. In part (a), we determined that since $R \equiv rup - r(p-1) \quad (\mod{p(p-1)})$, meaning that

\begin{align*}
    R &= rup - r(p - 1) + ap(p-1)
\end{align*}

for some integer $a$. We can manipulate this equation as follows

\begin{align*}
    R &= rup - r(p - 1) + ap(p - 1) \\
    &= rup - rp + r + ap(p - 1) \\
    &= r + p(a(p-1) + ru - r)
\end{align*}

Where $a(p-1) + ru - r$ is an integer. When we compute $R \quad (\mod{p})$, we see that

\begin{align*}
    R &\equiv r + p(a(p - 1) + ru - r) \quad (\mod{p}) \\
    &\equiv r \quad (\mod{p})
\end{align*}

We can remove $p(a(p - 1) + ru - r)$ because it is an integer multiple of $p$, and we are performing arithmetic modulo $p$. Therefore, we can see that $R \equiv r \quad (\mod{p})$. From this assertion, we can also see that

\begin{align*}
    R^S \equiv r^S \quad (\mod{p})
\end{align*}

By definition, $S \equiv su \quad (\mod{p - 1})$. This means that $S = su + j(p - 1)$ for some integer $j$. Knowing this, we can re-write $R^S \quad (\mod{p})$ as

\begin{align*}
    R^S &\equiv r^S \quad (\mod{p}) \\ 
    &\equiv r^{su + j(p - 1)} \quad (\mod{p}) \\
    &\equiv (r^{su}) (r^{p - 1})^j \quad (\mod{p})
\end{align*}

From the problem description, we can see that $r \equiv g^k \quad (\mod{p})$. Knowing this, we can manipulate $R^S \quad (\mod{p})$ even further to obtain

\begin{align*}
    R^S &\equiv (r^{su}) (r^{p - 1})^j \quad (\mod{p}) \\
    &\equiv (r^{su}) ((g^k)^{p - 1})^j \quad (\mod{p}) \\
    &\equiv (r^{su}) (g^{p - 1})^{jk} \quad (\mod{p})
\end{align*}

Like in part (a), since $p$ is a prime and $g$ is a primitive root of $p$, then by Fermat's Little Theorem, $g^{p - 1} \equiv 1 \quad (\mod{p})$. Using this information, we can modify $R^S$ one last time to obtain

\begin{align*}
    R^S &\equiv (r^{su}) (g^{p - 1})^{jk} \quad (\mod{p}) \\
    &\equiv (r^{su}) (1)^{jk} \quad (\mod{p}) \\
    &\equiv r^{su} \quad (\mod{p})
\end{align*}

Thus, we have proven that $R^S \equiv r^{su} \quad (\mod{p})$.
\\

\item % part (c)
Lastly, we must prove that ($R$, $S$) is a valid signature to the message $M'$, where $M'$ is an arbitrary message of Eve's choice. In other words, given a valid signature ($r$, $s$) to a message $M$, we must show that Eve can generate a valid signature ($R$, $S$) to \textit{any} message $M'$ so that it appears that the message is coming from Alice. \\

By definition, ($r$, $s$) is a valid signature for a message $M$ if and only if 

\begin{align*}
    y^rr^s \equiv g^{H(M)} \quad (\mod{p})
\end{align*}

Therefore, in order for ($R$, $S$) to be a valid signature for $M'$, it must be true that

\begin{align*}
    y^RR^S \equiv g^{H(M')} \quad (\mod{p})
\end{align*}

In part (a) we proved that 

\begin{align*}
    y^R \equiv y^{ru} \quad (\mod{p})
\end{align*}

In part (b), we proved that

\begin{align*}
    R^S \equiv r^{su} \quad (\mod{p})
\end{align*}

Therefore, with these two statements we can see that

\begin{align*}
    y^RR^S \equiv y^{ru}r^{su} \quad (\mod{p})
\end{align*}

Therefore ($R$, $S$) is a valid signature for $M'$, if and only if 

\begin{align*}
    y^{ru}r^{su} \equiv g^{H(M')} \quad (\mod{p})
\end{align*}

By definition $H(M') \equiv H(M)u \quad (\mod{p - 1})$. Therefore,

\begin{align*}
    H(M') = H(M)u + z(p - 1)
\end{align*}

for some integer $z$. With this knowledge, we can show that

\begin{align*}
    g^{H(M')} &\equiv g^{H(M)u + z(p - 1)} \quad (\mod{p}) \\
    &\equiv g^{H(M)u} (g^{p - 1})^z \quad (\mod{p})
\end{align*}

Once again, since $g$ is a primitive root of $p$, then $g^{p - 1} \equiv 1 \quad (\mod{p})$ by Fermat's Little Theorem. Knowing this, we can manipulate $g^{H(M')} \quad (\mod{p})$ even further to obtain

\begin{align*}
    g^{H(M')}  &\equiv g^{H(M)u} (g^{p - 1})^z \quad (\mod{p}) \\
    &\equiv g^{H(M)u} (1)^z \quad (\mod{p}) \\
    &\equiv g^{H(M)u} \quad (\mod{p})
\end{align*}

With the knowledge that $g^{H(M')} \equiv g^{H(M)u} \quad (\mod{p})$, we can now say that ($R$, $S$) is a valid signature for $M'$ if and only if

\begin{align*}
    y^{ru}r^{su} \equiv g^{H(M)u} \quad (\mod{p})
\end{align*}

We can use exponent rules on the left side of the congruence to obtain

\begin{align*}
    (y^rr^s)^u \equiv (g^{H(M)})^u \quad (\mod{p})
\end{align*}

Therefore, we can see that in this case ($R$, $S$) will be a valid signature for $M'$ if and only if $y^rr^s \equiv g^{H(M)} \quad (\mod{p})$. We know that this statement is true because ($r$, $s$) is a valid signature for $M$. By definition, this can only be true if $y^rr^s \equiv g^{H(M)} \quad (\mod{p})$. Therefore, since we know that

\begin{align*}
    y^rr^s \equiv g^{H(M)} \quad (\mod{p})
\end{align*}

is true, then that means 

\begin{align*}
    (y^rr^s)^u \equiv (g^{H(M)})^u \quad (\mod{p})
\end{align*}

This ultimately means that

\begin{align*}
    y^RR^S \equiv g^{H(M')} \quad (\mod{p})
\end{align*}

Therefore, ($R$, $S$) is a valid signature for $M'$, and we have proven what was asked of us.

\end{enumerate}
\end{enumerate}
\end{document}
