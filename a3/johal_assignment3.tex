\documentclass[11pt]{article}

\usepackage{amsthm,amsmath,amssymb,amsfonts}
\usepackage[margin=1in]{geometry}

\theoremstyle{definition}
\newtheorem*{solution}{Solution}
\newcounter{problem}

\newcommand{\CBCMAC}{\text{CBC-MAC}}
\newcommand{\AHMAC}{\mathrm{AHMAC}}
\newcommand{\PHMAC}{\mathrm{PHMAC}}
\renewcommand{\pmod}[1]{\mbox{\ $(\ensuremath{\operatorname{mod}}\ {#1})$}}
\newcommand{\Z}{\mathbb{Z}}
\providecommand{\Leg}[2]{\genfrac{(}{)}{}{}{#1}{#2}}


\begin{document}

\begin{center}
{\bf \Large CPSC 418 / MATH 318 --- Introduction to Cryptography

ASSIGNMENT 3 }
\end{center}

\hrule 	

\textbf{Name:} Harjee Johal \\
\textbf{Student ID:} 30000668

\medskip \hrule

\begin{enumerate} \itemsep 20pt

\stepcounter{problem}
\item[] \textbf{Problem \theproblem} --- Flawed MAC designs, 13 marks

\begin{enumerate}

\item The message $M_2$ is of the form $M_2 = M_1 || X$, where $X$ is an arbitrary $n$-bit block. From
the question description we also know that $M_1 = P_1 || P_2 || \ldots || P_L$, where $P_1, P_2, \ldots, P_L$ are
each $n$-bit blocks. 

\item % part (b)
\end{enumerate}

\newpage

\stepcounter{problem}
\item[] \textbf{Problem \theproblem} --- Fast RSA decryption using Chinese remaindering, 8
    marks)

We are told that $d_p \equiv d \quad \mod{p - 1}$. This means that $d_p$ can be written in the form: $d_p = d + j(p-1)$, where $j$ is an integer. We are also told that $d_q \equiv d \quad \mod{q - 1}$. This means that $d_q$ can be written in the form: $d_q = d + k(q-1)$, where $k$ is an integer. \\ 

Next, we are told that $M_p \equiv C^{d_p} \quad \mod{p}$. Since we know that $d_p = d + j(p-1)$, that means that we can re-write this as: $M_p \equiv C^{d + j(p-1)} \quad \mod{p}$. We can manipulate this using power rules to obtain:

\begin{align*}
    &C^{d + j(p-1)} \quad \mod{p} \\
    \equiv &C^d C^{j(p-1)} \quad \mod{p} \\
    \equiv &C^d (C^{p-1})^j \quad \mod{p}
\end{align*}

Since $p$ is a prime, that means that $\phi{(p)} = p - 1$. Furthermore, since we're told that $gcd(n, C) = 1$, that means that $C$ and $n$ share no prime factors. Since $n$'s prime factors are $p$ and $q$, that means that $gcd(p, C)$ and $gcd(q, C)$ must also be 1. Since $\phi{(p)} = p - 1$ and $gcd(p, C) = 1$, we can apply Euler's theorem on $C^{p - 1}$, meaning that $C^{p - 1} \equiv 1 \quad \mod{p}$. Using this, we can further simplify $M_p$:

\begin{align*}
    &C^d (C^{p-1})^j \quad \mod{p} \\
    \equiv &C^d (1)^j \quad \mod{p} \\
    \equiv &C^d (1) \quad \mod{p} \\
    \equiv &C^d \quad \mod{p}
\end{align*}

Thus, we can see that $M_p \equiv C^d \quad \mod{p}$. That means that $M_p = C^d + sp$, where $s$ is an integer. \\

Furthermore, we are told that $M_q \equiv C^{d_q} \quad \mod{q}$. Since we know that $d_q = d + k(q-1)$, that means that we can re-write this as: $M_q \equiv C^{d + k(q-1)} \quad \mod{q}$. We can manipulate this using power rules to obtain:

\begin{align*}
    &C^{d + k(q-1)} \quad \mod{q} \\
    \equiv &C^d C^{k(q-1)} \quad \mod{q} \\
    \equiv &C^d (C^{q-1})^k \quad \mod{q}
\end{align*}

Since $q$ is a prime, that means that $\phi{(q)} = q - 1$. Earlier, we demonstrated that since $gcd(n, C) = 1$, then $gcd(q, C)$ must also equal 1. Since $\phi{(q)} = q - 1$ and $gcd(q, C) = 1$, we can apply Euler's theorem on $C^{q - 1}$, meaning that $C^{q - 1} \equiv 1 \quad \mod{q}$. Using this, we can further simplify $M_q$:

\begin{align*}
    &C^d (C^{q-1})^k \quad \mod{q} \\
    \equiv &C^d (1)^k \quad \mod{q} \\
    \equiv &C^d (1) \quad \mod{q} \\
    \equiv &C^d \quad \mod{q}
\end{align*}


Thus, we can see that $M_q \equiv C^d \quad \mod{q}$. That means that $M_q = C^d + tq$, where $t$ is an integer. \\

Lastly, we are told that $M \equiv pxM_q + qyM_p \quad \mod{n}$. We can substitute $M_p$ with $C^d + sp$ and $M_q$ with $C^d + tq$ to obtain:

\begin{align*}
    M &\equiv pxM_q + qyM_p \quad \mod{n} \\
    &\equiv px(C^d + tq) + qy(C^d + sp) \quad \mod{n} \\
    &\equiv pxC^d + pqtx + qyC^d + pqsy \quad \mod{n} \\
    &\equiv C^d(px + qy) + pq(tx + sy) \quad \mod{n}
\end{align*}

Where $tx + sy$ is an integer. We know that $n = pq$, and from step 3 of the algorithm we know that $px + qy = 1$. Using this information, we obtain:

\begin{align*}
    M &\equiv C^d(px + qy) + pq(tx + sy) \quad \mod{n} \\
    &\equiv C^d + n(tx + sy) \quad \mod{n} \\
    &\equiv C^d \quad \mod{n}
\end{align*}

We can remove $n(tx + sy)$ from the expression since it's a multiple of $n$, and the modular arithmetic is being done modulus $n$. Therefore, we obtain $M \equiv C^d \quad \mod{n} \equiv M^{ed} \quad \mod{n}$. In RSA, integers $e$ and $d$ are determined so that $ed \equiv 1 \quad \mod{n}$. Therefore, that means that $M \equiv M^{ed} \quad \mod{n} \equiv M^1 \quad \mod{n} \equiv M \quad \mod{n}$. Therefore, the $M$ obtained using this method of decryption is the same as the $M$ determined during "normal" RSA decryption.

\newpage

\stepcounter{problem}
\item[] \textbf{Problem \theproblem} ---  RSA primes too close together, 18 marks)


\begin{enumerate}
\item % part (a)
We are told that $y > 0$. We are also told that $x + y = p$. This means that $y = p - x$. Since $y > 0$, then that means that $p - x > 0$ as well. Since $p - x > 0$, then $p > x$. \\

We are told that $n = x^2 - y^2$. This means that $y^2 = x^2 - n$. Since $y > 0$, that means that $y^2 > 0$. Since $y^2 = x^2 - n$, and $y^2 > 0$, then that means $x^2 - n > 0$ as well. If $x^2 - n > 0$, then $x^2 > n$, which means that $x > \sqrt{n}$. \\

From this, we can see that $p > x$ and that $x > \sqrt{n}$. Thus, we can see that $p > x > \sqrt{n}$.


\item  % part (b)
We're told that in the Fermat factorization algorithm there is a \textit{while} loop that computes both $a = a + 1$ and $b = \sqrt{a^2 - n}$. This loop continues while the computed value of $b$ isn't an integer. \\

We know that $n = x^2 - y^2$. Therefore, we can re-write this equation as: $b = \sqrt{a^2 - (x^2 - y^2)}$. When $a = x$, then we get:

\begin{align*}
    b &= \sqrt{a^2 + y^2 - x^2} \\
    b &= \sqrt{(x)^2 + y^2 - x^2} \\
    b &= \sqrt{y^2} \\
    b &= y
\end{align*}

Thus, when $a = x$, then $b = y$. Since by definition $y$ is an integer, that means that the \textit{while} loop will terminate when $a = x$. \\

We're also told that the algorithm outputs $a - b$ once it's completed. Since the \textit{while} loop terminates when $a = x$, and $b = y$ when $a = x$, then that means $a - b = x - y$. By definition, $q = x - y$, meaning that when this algorithm terminates, it outputs $q$. \\

We can also show that the \textit{while} loop won't terminate for a value of $a$ less than $x$ using a proof by contradiction. Suppose that the loop does terminate for some integer $a < x$. Then that means that for that value of $a$, we get $b = \sqrt{a^2 - n}$, where $b$ is an integer. We can re-arrange the equation $b = \sqrt{a^2 - n}$ to obtain:

\begin{align*}
    b &= \sqrt{a^2 - n} \\
    b^2 &= a^2 - n \\
    n &= a^2 - b^2 \\
    n &= (a + b)(a - b)
\end{align*}

From this, we can see that $n = (a + b)(a - b)$. Since $n = pq$ and $n > p > q > 0$, then that means $p = a + b$ and $q = a - b$. From these two equations, we can see that $a = \frac{p + q}{2}$. By definition, $x = \frac{p + q}{2}$. Therefore, that would mean that $a = x$. However, we defined $a$ such that $a < x$. This is a contradiction. Therefore, this shows that the loop does not terminate for any value of $a < x$.

\item  % part (c)
The value of $a$ is initialized as $a = \lceil \sqrt{n} \rceil$. During the first iteration of the loop, the algorithm first performs $a = a + 1$, and then computes $b = \sqrt{a^2 - n}$. It continues to do this until it finally reaches $a = x$. This means that the first iteration of the loop is performed with $a = \lceil \sqrt{n} \rceil + 1$, and the last iteration of the loop is done with $a = x$. The number of iterations between the first and last values of $a$ is thus $x - \lceil \sqrt{n} \rceil$. The last iteration of the loop is after $a = x$. The condition at the top of the \textit{while} loop is checked one last time. Since $b$ is an integer when $a = x$, that means that the condition at the top of the \textit{while} loop will not be satisfied on the last iteration, meaning that the \textit{while} loop will be skipped. However, this iteration is still counted. Therefore, there are $x - \lceil \sqrt{n} \rceil$ iterations where the loop is entered, and $1$ iteration where the loop is not entered. Therefore, there are $x - \lceil \sqrt{n} \rceil + 1$ iterations overall.

\item  % part (d)
We are asked to prove that $x - \lceil \sqrt{n} \rceil < \frac{y^2}{2\sqrt{n}}$. We know that $n = x^2 - y^2$. We can rearrange this to get $y^2 = x^2 - n = (x + \sqrt{n})(x - \sqrt{n})$. We can re-arrange this equation to get: $(x - \sqrt{n}) = \frac{y^2}{x + \sqrt{n}}$. \\

In part (a), we showed that $x > \sqrt{n}$. From this, we can see that $x + \sqrt{n} > 2\sqrt{n}$. Since $2\sqrt{n} < x + \sqrt{n}$, that means that $\frac{y^2}{2\sqrt{n}} > \frac{y^2}{x + \sqrt{n}}$. Since $(x - \sqrt{n}) = \frac{y^2}{x + \sqrt{n}}$, that means that $(x - \sqrt{n}) < \frac{y^2}{2\sqrt{n}}$. \\

By definition, $\lceil \sqrt{n} \rceil \geq \sqrt{n}$. Since $\lceil \sqrt{n} \rceil \geq \sqrt{n}$, then that means $x - \sqrt{n} \geq x - \lceil \sqrt{n} \rceil$. We can use this to show that since $(x - \sqrt{n}) < \frac{y^2}{2\sqrt{n}}$, that means $(x - \lceil \sqrt{n} \rceil) < \frac{y^2}{2\sqrt{n}}$. Thus, we have proven that $x - \lceil \sqrt{n} \rceil < \frac{y^2}{2\sqrt{n}}$.

\item  % part (e)
Suppose that $p - q < 2B\sqrt[4]{n}$. We can re-arrange this to obtain:

\begin{align*}
    \frac{p - q}{2} &< B\sqrt[4]{n} \\
    y &< B\sqrt[4]{n} \quad \quad (y = \frac{p - q}{2}) \\
    y^2 &< B^2\sqrt{n} \\
    \frac{y^2}{\sqrt{n}} &< B^2 \\
    \frac{y^2}{2\sqrt{n}} &< \frac{B^2}{2}
\end{align*}

Thus, we can see that $\frac{y^2}{2\sqrt{n}} < \frac{B^2}{2}$. In part (d), we showed that $x - \lceil \sqrt{n} \rceil < \frac{y^2}{2\sqrt{n}}$. Using this statement, we can see that $x - \lceil \sqrt{n} \rceil < \frac{B^2}{2}$. Since $x - \lceil \sqrt{n} \rceil < \frac{B^2}{2}$, then that means that $x - \lceil \sqrt{n} \rceil + 1 < \frac{B^2}{2} + 1$. We showed in part (c) that the number of loops iterations performed by Fermat's Factorization algorithm to factor $n$ is also $x - \lceil \sqrt{n} \rceil + 1$. Therefore, we can see that algorithm factors $n$ after at most $\frac{B^2}{2} + 1$ iterations. 

\end{enumerate}

\newpage

\stepcounter{problem}
\item[] \textbf{Problem \theproblem} -- The El Gamal public key cryptosystem is not
    semantically secure, 12 marks

We are asked to prove that El Gamal is not semantically secure. We do this by proving 6 assertions. The assertions are as follows: \\

\textbf{\textit{Assertion 1:}} If $\Leg{y}{p} = 1$ and $\Leg{C_2}{p} = 1$, then $C = E(M_1)$:

Since $C_2 \equiv My^k \mod{p}$, this implies that $\Leg{C_2}{p} = \Leg{My^k}{p} = \Leg{M}{p} \Leg{y}{p}^k$. Since $\Leg{y}{p} = 1$ and $\Leg{C_2}{p} = 1$, we can substitute them into the equation to get: $1 = \Leg{M}{p} (1)^k = \Leg{M}{p}$. Since $\Leg{M}{p} = 1$, then that means that $M$ is a quadratic residue modulo $p$. We are told in the question that $M_1$ is a quadratic residue modulo $p$, and $M_2$ is not a quadratic residue modulo $p$. Therefore since the $M$ in $C_2 \equiv My^k \mod{p}$ is a quadratic residue modulo $p$, it must be $M_1$, meaning that in this case $C = E(M_1)$. Thus the assertion is true. \\ \\


\textbf{\textit{Assertion 2:}} If $\Leg{y}{p} = 1$ and $\Leg{C_2}{p} = -1$, then $C = E(M_2)$:

Since $C_2 \equiv My^k \mod{p}$, this implies that $\Leg{C_2}{p} = \Leg{My^k}{p} = \Leg{M}{p} \Leg{y}{p}^k$. Since $\Leg{y}{p} = 1$ and $\Leg{C_2}{p} = -1$, we can substitute them into the equation to get: $-1 = \Leg{M}{p} (1)^k = \Leg{M}{p}$. Since $\Leg{M}{p} = -1$, then that means that $M$ is not a quadratic residue modulo $p$. We are told in the question that $M_1$ is a quadratic residue modulo $p$, and $M_2$ is not a quadratic residue modulo $p$. Therefore since the $M$ in $C_2 \equiv My^k \mod{p}$ is not a quadratic residue modulo $p$, it must be $M_2$, meaning that in this case $C = E(M_2)$. Thus the assertion is true. \\ \\


\textbf{\textit{Assertion 3:}} If $\Leg{y}{p} = -1$ and $\Leg{C_1}{p} = 1$ and $\Leg{C_2}{p} = 1$, then $C = E(M_1)$:

Since $y \equiv g^x \mod{p}$, then that means that $\Leg{y}{p} = \Leg{g}{p}^x$. Since we know that $\Leg{y}{p} = -1$, that means that $-1 = \Leg{g}{p}^x$. From this, we can see that $\Leg{g}{p} = -1$, and $x$ must be an odd integer. Next, we see that $C_1 \equiv g^k \mod{p}$, meaning that $\Leg{C_1}{p} = \Leg{g}{p}^k$. We know that $\Leg{C_1}{p} = 1$ and $\Leg{g}{p} = -1$, meaning that $1 = (-1)^k$. Therefore, $k$ must be an even number. \\

Next, we can see that since $C_2 \equiv My^k \mod{p}$, then $\Leg{C_2}{p} = \Leg{M}{p} \Leg{y}{p}^k$. We know that $\Leg{C_2}{p} = 1$ and that $\Leg{y}{p} = -1$ and that $k$ is an even number. Therefore, we can see that $1 = \Leg{M}{p} (-1)^{even} = \Leg{M}{p}$. Since $\Leg{M}{p} = 1$, then that means that $M$ is a quadratic residue modulo $p$. We know that $M_1$ is a quadratic residue modulo $p$, and $M_2$ is not a quadratic residue modulo $p$. Therefore since the $M$ in $C_2 \equiv My^k \mod{p}$ is a quadratic residue modulo $p$, it must be $M_1$, meaning that in this case $C = E(M_1)$. Thus the assertion is true. \\ \\


\textbf{\textit{Assertion 4:}} If $\Leg{y}{p} = -1$ and $\Leg{C_1}{p} = 1$ and $\Leg{C_2}{p} = -1$, then $C = E(M_2)$:

Since $y \equiv g^x \mod{p}$, then that means that $\Leg{y}{p} = \Leg{g}{p}^x$. Since we know that $\Leg{y}{p} = -1$, that means that $-1 = \Leg{g}{p}^x$. From this, we can see that $\Leg{g}{p} = -1$, and $x$ must be an odd integer. Next, we see that $C_1 \equiv g^k \mod{p}$, meaning that $\Leg{C_1}{p} = \Leg{g}{p}^k$. We know that $\Leg{C_1}{p} = 1$ and $\Leg{g}{p} = -1$, meaning that $1 = (-1)^k$. Therefore, $k$ must be an even number. \\

Next, we can see that since $C_2 \equiv My^k \mod{p}$, then $\Leg{C_2}{p} = \Leg{M}{p} \Leg{y}{p}^k$. We know that $\Leg{C_2}{p} = -1$ and that $\Leg{y}{p} = -1$ and that $k$ is an even number. Therefore, we can see that $-1 = \Leg{M}{p} (-1)^{even} = \Leg{M}{p}$. Since $\Leg{M}{p} = -1$, then that means that $M$ is not a quadratic residue modulo $p$. We know that $M_1$ is a quadratic residue modulo $p$, and $M_2$ is not a quadratic residue modulo $p$. Therefore since the $M$ in $C_2 \equiv My^k \mod{p}$ is not a quadratic residue modulo $p$, it must be $M_2$, meaning that in this case $C = E(M_2)$. Thus the assertion is true. \\ \\


\textbf{\textit{Assertion 5:}} If $\Leg{y}{p} = -1$ and $\Leg{C_1}{p} = -1$ and $\Leg{C_2}{p} = 1$, then $C = E(M_2)$:

Since $y \equiv g^x \mod{p}$, then that means that $\Leg{y}{p} = \Leg{g}{p}^x$. Since we know that $\Leg{y}{p} = -1$, that means that $-1 = \Leg{g}{p}^x$. From this, we can see that $\Leg{g}{p} = -1$, and $x$ must be an odd integer. Next, we see that $C_1 \equiv g^k \mod{p}$, meaning that $\Leg{C_1}{p} = \Leg{g}{p}^k$. We know that $\Leg{C_1}{p} = -1$ and $\Leg{g}{p} = -1$, meaning that $-1 = (-1)^k$. Therefore, $k$ must be an odd number. \\

Next, we can see that since $C_2 \equiv My^k \mod{p}$, then $\Leg{C_2}{p} = \Leg{M}{p} \Leg{y}{p}^k$. We know that $\Leg{C_2}{p} = 1$ and that $\Leg{y}{p} = -1$ and that $k$ is an odd number. Therefore, we can see that $1 = \Leg{M}{p} (-1)^{odd} = -\Leg{M}{p}$, meaning that $\Leg{M}{p} = -1$. Since $\Leg{M}{p} = -1$, then that means that $M$ is not a quadratic residue modulo $p$. We know that $M_1$ is a quadratic residue modulo $p$, and $M_2$ is not a quadratic residue modulo $p$. Therefore since the $M$ in $C_2 \equiv My^k \mod{p}$ is not a quadratic residue modulo $p$, it must be $M_2$, meaning that in this case $C = E(M_2)$. Thus the assertion is true. \\ \\


\textbf{\textit{Assertion 6:}} If $\Leg{y}{p} = -1$ and $\Leg{C_1}{p} = -1$ and $\Leg{C_2}{p} = -1$, then $C = E(M_1)$:

Since $y \equiv g^x \mod{p}$, then that means that $\Leg{y}{p} = \Leg{g}{p}^x$. Since we know that $\Leg{y}{p} = -1$, that means that $-1 = \Leg{g}{p}^x$. From this, we can see that $\Leg{g}{p} = -1$, and $x$ must be an odd integer. Next, we see that $C_1 \equiv g^k \mod{p}$, meaning that $\Leg{C_1}{p} = \Leg{g}{p}^k$. We know that $\Leg{C_1}{p} = -1$ and $\Leg{g}{p} = -1$, meaning that $-1 = (-1)^k$. Therefore, $k$ must be an odd number. \\

Next, we can see that since $C_2 \equiv My^k \mod{p}$, then $\Leg{C_2}{p} = \Leg{M}{p} \Leg{y}{p}^k$. We know that $\Leg{C_2}{p} = -1$ and that $\Leg{y}{p} = -1$ and that $k$ is an odd number. Therefore, we can see that $-1 = \Leg{M}{p} (-1)^{odd} = -\Leg{M}{p}$, meaning that $\Leg{M}{p} = 1$. Since $\Leg{M}{p} = 1$, then that means that $M$ is a quadratic residue modulo $p$. We know that $M_1$ is a quadratic residue modulo $p$, and $M_2$ is not a quadratic residue modulo $p$. Therefore since the $M$ in $C_2 \equiv My^k \mod{p}$ is a quadratic residue modulo $p$, it must be $M_1$, meaning that in this case $C = E(M_1)$. Thus the assertion is true. \\

We've proved that all of Mallory's assertions are true. Therefore, El Gamal is not semantically secure.

\newpage


\stepcounter{problem}
\item[] \textbf{Problem \theproblem} --- An IND-CPA, but not IND-CCA secure version of RSA, 12
    marks
    
    We are asked to show that the version of RSA specified in this question is not IND-CCA secure. We start by choosing two different plaintexts, $M_1$ and $M_2$, and receive a ciphertext $C$ that is an encryption of one of them. The ciphertext can be represented as:
    
    \begin{align*}
        C = (s||t) = (r^e (mod \quad n) || H(r) \oplus M_i)
    \end{align*}
    
    where i = 1 or i = 2. The value $r$ is a random $k$-bit value such that $r < n$. In this case, $n$ also has $k$-bits. $H$ is a public random function that maps $\{0, 1\}^k$ to $\{0, 1\}^m$, where $m$ is the bit-length of the message being encrypted. \\
    
    We then compute a new ciphertext, $C'$ from $C$, such that:
    
    \begin{align*}
        C' = (s||t')
    \end{align*}
    
    where $t' = t \oplus M_1$. From here, we have two cases: \\
    
    \textbf{\textit{Case 1:}} $C$ is an encryption of $M_1$.
    
    In the question, we are told that decryption is done via $M \equiv H(s^d (mod \quad n)) \oplus t$. The decryption of $C'$ would therefore be: $M \equiv H(s^d (mod \quad n)) \oplus t'$. Since $t' = t \oplus M_1$, and in this case we know that $t = H(r) \oplus M_1$, that means that $t' = H(r) \oplus M_1 \oplus M_1$. A number XOR'd with itself equals zero, meaning that $t' = H(r) \oplus 0 = H(r)$. Therefore, the decryption of $C'$ can be simplified to: 
    
    \begin{align*}
        M &\equiv H(s^d (mod \quad n)) \oplus t' \\
        &\equiv H(s^d (mod \quad n)) \oplus H(r)
    \end{align*}
    
    We know that $s = r^e$, meaning $s^d = r^{ed}$. By the definition of $e$ and $d$, we know that $ed = 1 + k\phi{(n)}$, meaning that $r^{ed} = r^{1 + k\phi{(n)}} = r (r^{\phi{(n)}})$. We can use Euler's theorem to show that $r^{\phi{(n)}} \equiv 1 \mod{n}$ (the probability of $gcd(r, n) \neq 1$ is very low). Thus, we can see that $r^{ed} \equiv r \mod{n}$. Therefore, $H(s^d (mod \quad n)) = H(r)$. Thus, we can once again modify the decryption process to see that:
    
    \begin{align*}
        M &\equiv H(s^d (mod \quad n)) \oplus H(r) \\
        &= H(r) \oplus H(r) \\
        &= 0
    \end{align*}
    
    Therefore, when $C$ is an encryption of $M_1$, then the decryption of $C'$ is 0. \\ \\
    
    
    \textbf{\textit{Case 2:}} $C$ is an encryption of $M_2$.
    
    In the question, we are told that decryption is done via $M \equiv H(s^d (mod \quad n)) \oplus t$. The decryption of $C'$ would therefore be: $M \equiv H(s^d (mod \quad n)) \oplus t'$. Since $t' = t \oplus M_1$, and in this case we know that $t = H(r) \oplus M_2$, that means that $t' = H(r) \oplus M_1 \oplus M_2$. Therefore, the decryption of $C'$ can be re-written as: 
    
    \begin{align*}
        M &\equiv H(s^d (mod \quad n)) \oplus t' \\
        &\equiv H(s^d (mod \quad n)) \oplus H(r) \oplus M_1 \oplus M_2
    \end{align*}
    
    From the previous case, we showed that $H(s^d (mod \quad n)) = H(r)$. Thus, we can once again modify the decryption process to see that:
    
    \begin{align*}
        M &\equiv H(s^d (mod \quad n)) \oplus H(r) \oplus M_1 \oplus M_2 \\
        &= H(r) \oplus H(r) \oplus M_1 \oplus M_2 \\
        &= M_1 \oplus M_2
    \end{align*}
    
    Therefore, when $C$ is an encryption of $M_2$, then the decryption of $C'$ is $M_1 \oplus M_2$. Mallory can easily compute this value, since she selects both $M_1$ and $M_2$. \\
    
    From this, we can see that Mallory has a method to easily identify whether $C$ is the ciphertext of $M_1$ or $M_2$ based on the decryption of $C'$. If the decryption of $C'$ is 0, then $C$ is an encryption of $M_1$. Otherwise, if the decryption of $C'$ is $M_1 \oplus M_2$, then $C$ is an encryption of $M_2$.
    

\newpage

\stepcounter{problem}
\item[]  \textbf{Problem \theproblem} --- An attack on RSA with small decryption exponent, 25
    marks

\begin{enumerate}

\item % part (a)

\item % part (b)

\item % part (c)

\item % part (d)

\item % part (e)

\item % part (f)

\end{enumerate}

\newpage

\stepcounter{problem}
\item[] \textbf{Problem \theproblem} --- Universal forgery attack on the El Gamal signature
    scheme, 12 marks)

\begin{enumerate}

\item % part (a)

\item % part (b)

\item % part (c)

\end{enumerate}

\newpage

\stepcounter{problem} \stepcounter{problem}

\item[] \textbf{Problem \theproblem} --- Columnar transposition cryptanalysis, 10 marks

\end{enumerate}
\end{document}
